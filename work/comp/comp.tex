\documentclass[12pt]{article}
\usepackage[margin=1in]{geometry}
\usepackage{amssymb}
\usepackage{amsmath}
\usepackage{url}
\usepackage{bm}
\usepackage{color}
\usepackage{graphicx}
\usepackage{cite}

\newcommand{\red}[1]{{\color{red}{#1}}}
\newcommand{\blue}[1]{{\color{blue}{#1}}}
\newcommand{\ket}[1]{\left| #1 \right>}
\newcommand{\bra}[1]{\left< #1 \right|}
\newcommand{\braket}[2]{\left< #1 | #2 \right>}
\newcommand{\ketbra}[2]{\left| #1 \right> \left< #2 \right|}
\newcommand{\expect}[1]{\left< #1 \right>}
\newcommand{\fpij}{f_p(r_{ij})}
\newcommand{\vpij}{v_p(r_{ij})}
\newcommand{\Opij}{\mathcal{O}_{ij}^p}
\newcommand{\fOpij}{\sum\limits_{i<j}\sum\limits_p \fpij\Opij}
\newcommand{\fqkl}{f_q(r_{kl})}
\newcommand{\Oqkl}{\mathcal{O}_{kl}^q}
\newcommand{\fOqkl}{\sum\limits_{k<l}\sum\limits_q \fqkl\Oqkl}
\newcommand{\fOqklip}{\sum\limits_{k<l,\mathrm{ip}}\sum\limits_q \fqkl\Oqkl}
\newcommand{\fOqklquad}{\sum_{\substack{k<l\\ij \ne kl}}\sum\limits_q \fqkl\Oqkl}
\newcommand{\f}[2]{f_{#1}(r_{#2})}
\renewcommand{\O}[2]{\mathcal{O}_{#2}^{#1}}
\newcommand{\fO}[2]{\sum\limits_{#1} f_{#1}(r_{#2})\mathcal{O}_{#2}^{#1}}
\newcommand{\R}{\mathbf{R}}
\newcommand{\dt}{\Delta\tau}
\newcommand{\ti}{\bm{\tau}_i}
\newcommand{\tj}{\bm{\tau}_j}
\newcommand{\si}{\bm{\sigma}_i}
\newcommand{\sj}{\bm{\sigma}_j}

\title{{\large Written Comprehensive Exam:}\\Improved Trial Wave Function for Quantum Monte Carlo Calculations of Nuclear Systems}
\author{Cody L. Petrie \\
Advisor: Kevin Schmidt}

\begin{document}
\maketitle

\begin{abstract}
We have done Auxiliary Field Diffusion Monte Carlo calculations to determine the energy of $^{4}$He and $^{16}$O and the energy per nucleon of symmetric nuclear matter (SNM) with 28 particles in a periodic box with density $\rho=0.16$fm$^{-3}$. These calculations have been done with two new types of correlations in the trial wave function that include quadratic terms in the expansion of the correlation term. These results have been compared to previous results done with linear correlations.
\end{abstract}

\section{Background and Motivation}
Nuclear physics sheds light on the extremes. From the structure and processes of atomic nuclei and hypernuclei to the formation and structure of some of the largest structures in the universe, neutron stars. One of the largest obstacles to these regimes is our lack of understanding of the interactions between nucleons and how to most effectively apply that knowledge to calculate properties of nuclear systems. Despite the difficulty, science has been making continuous steps toward that understanding. In 1935 Hideki Yukawa proposed the idea that the nuclear interaction, called the strong force, was governed by quanta or exchange particles called pions \cite{yukawa1935}. From this idea came the Yukawa potential, which is still used in modified form in many nuclear models today. The range of the force proposed by Yukawa was based on the mass of the exchange particle, and the strength was based only on the distance separating the particles. Today we often use potentials that depend on the separation distance between particles, but also their relative spins and isospins. These interactions can be quite complicated making a true understanding of the strong force difficult to achieve.

Currently it is believed that Quantum Chromodynamics (QCD) is the most correct theory to describe the strong force. However, due to asymptotic freedom, at low energies this theory becomes quite difficult to use and so other, approximate methods are often used to study the strong interaction. We use Quantum Monte Carlo methods to investigate different aspects of the strong interaction.

Many appriximate methods exist to solve the nuclear many body problem. Some of these include ??????????

Continuing to better understand the interactions between nuclei will advance our understanding of many important processes in the universe.


\documentclass[12pt]{extarticle}
\usepackage[margin=1in]{geometry}
\usepackage{amsmath}
\usepackage{url}
\usepackage{color}
\usepackage{graphicx}

% my commands
\newcommand{\red}[1]{{\color{red}{#1}}}
\newcommand{\R}{\mathbf{R}} % walker
\newcommand{\ket}[1]{\left| #1 \right>}
\newcommand{\bra}[1]{\left< #1 \right|}
\newcommand{\braket}[2]{\left< #1 | #2 \right>}

\title{Other methods used to solve the nuclear many-body problem}
\author{Cody L. Petrie}

\begin{document}
\maketitle

% include sections here
\section{Hartree-Fock}
The general starting place for the HF method is to start with a Slater determiant. The problem is then similar to the minimization problem in classical mechanics where a Lagrangian is formed, which then defines an action, $\mathcal{S}$, which is then minimized with respect to the individual single particle states
\begin{equation}
   \frac{\partial\mathcal{L}}{\partial \psi^*_i} - \sum\limits_{\alpha=1}^3 \frac{\partial}{\partial x_\alpha} \frac{\partial \mathcal{L}}{\left(\frac{\partial\psi^*_i}{\partial x_\alpha}\right)} = 0
\end{equation}
This is written in terms of derivatives with respect to $\psi$ instead of the coordinates because the lagrangian is written as $\mathcal{L} = \mathcal{L}(\psi^*,\psi,\nabla\psi^*,\nabla\psi)$, like in QFT.

The constraint that
\begin{equation}
   C_i(\psi^*,\psi) = \int d^3r \psi^*_i(\mathbf{r})\psi_i(\mathbf{r}) = 1,
\end{equation}
can then be included in the usual way, similar to in classical mechanics text books
\begin{equation}
   \left[\frac{\delta\mathcal{S}}{\delta \psi^*_i} + \lambda_i\frac{\delta C_i}{\delta\psi^*_i}\right] = 0.
\end{equation}
This is then used to solve for what is called the Hartree-Fock equations
\begin{equation}
   \epsilon_i\phi_i(\mathbf{r}) = -\frac{\hbar^2}{2m_N}\nabla^2\phi(\mathbf{r}) + U^D(\mathbf{r})\phi_i(\mathbf{r}) - \int d^3\mathbf{r}'U^X(\mathbf{r},\mathbf{r}')\phi(\mathbf{r}'),
\end{equation}
where $U^D$ and $U^X$ are written in terms of the one-body density matrix
\begin{equation}
   \rho(\mathbf{r},\mathbf{r}') = \sum\limits_j \phi^*_j(\mathbf{r})\phi_j(\mathbf{r}').
\end{equation}

\section{No-Core Shell Model}
For a VERY brief explanation see \cite{lynn2013}, or for a more in depth review see \cite{navratil2009}.

The basic idea is similar to ours. They start from a Hamiltonian
\begin{equation}
   H = \sum\limits_{i=1}^A\frac{\mathbf{p}_i^2}{2m_N} + \sum\limits_{i<j}^A v_{ij} + \sum\limits_{i<j<k}^A V_{ijk}
\end{equation}
The wave function basis is a truncated HO basis. This allows for ``second quantization", which allows the shell model states to be used, and Navratil mentions that it also allows them to use ``single-nucleon coordinates" without losing translational invariance.

They can handle non-local potentials because they are using the HO basis. However, they can't handle sharp (e.g. strong short-range correlations) interactions and so they use an effectice $H$ formed by operating on $H$ with a unitary tranformation, which introduces errors.

It looks like they can handle up to $A$=16 nuclei, at least up until their 2009 review.

It sounds like they directly solve the eigenvalue problem in the various $j$ channels.

\section{Coupled Cluster}
For the specific references to this see \cite{lynn2013}.

It sounds like they do best with closed-shell nuclei where symmetries can reduce the complexity of the calculations. They have only ever reported energies for $^{16}$O, but did mention calculations for $^{40}$Ca. It must be a similar situation to us, $^{40}$Ca is calculable, but doesn't give very good accuracy.

It sounds like they do the same thing as NCSM, except that they try to obey size extensivity and size consistency.
\begin{enumerate}
   \item Size Extensivity: ``Only linked diagrams appear in the calculation of the expectation value of the energy." This has the do with the linear behaviour of the method with respect to particle number, but I don't really know other than that. Everybody is vague.
   \item Size Consistency: This refers to the energy of two non-interacting systems to be the sum of their two energies, $E(A+B)= E(A) + E(B)$.
\end{enumerate}

This is done by writting the wave function as being correlated by an exponential correlations operator
\begin{equation}
    \ket{\Psi} = \exp(T_{corr})\ket{\Psi_0},
\end{equation}
where the $\ket{\Psi_0}$ state is the single-particle state, usually something like a Slater determinant. The expectation value of the Hamiltonian can then be written in terms of the NON-HERMITIAN operator, $\exp(-T_{corr})H\exp(T_{corr})$
\begin{equation}
   \left<H\right> = \bra{\Psi_0}\exp(-T_{corr})H\exp(T_{corr})\ket{\Psi_0}.
\end{equation}

\section{Self-Consistent Green's Function}

\bibliographystyle{unsrt} % unsrt shows in order of citations
\bibliography{../../../../Dropbox/nuclear/papers/references.bib}

\end{document}


\section{Trial wave function}
The trial wave function is used in both the diffusion and branching algorithms and thus it is important that it be close to the ground state wave function of the system. Propagation in imaginary time removes the excited states from the trial wave function and so a larger overlap between the trial wave function and the ground state wave function means quicker convergence in imaginary time. One of the simplest many particle wave functions is an antisymmetrized product of single particle orbitals,
\begin{equation}
   \psi_{T} = \mathcal{A} \prod\limits_{i=1}^A \phi_i(\mathbf{r}_\alpha,s_\alpha) = \frac{1}{A!} \mathrm{det}~\phi_i(\mathbf{r}_\alpha,s_\alpha),
\end{equation}
where $\mathcal{A}$ is an antisymmetrization operator with $\mathcal{A}^2=\mathcal{A}$. This is often called a Slater determinant. This wave function represents the long range part of the interactions leaving out short range correlations. The simplest short range correlation that can be formed is a spin-isospin independent Jastrow-like correlation only dependent on space,
\begin{equation}
   \ket{\psi_T} = \prod\limits_{i<j}f(r_{ij}) \ket{\phi},
\end{equation}
where $r_{ij}$ are operators. %This does not include any spin-isospin correlations, and also is not cluster decomposable. A system that obeys cluster decomposition is a system that obeys locality. It manifests itself in the independence of systems that are separated in space. If the system of particles is split into two subsystems, $\Phi_A$ and $\Phi_B$, that are far apart, the wave function of the physical system would be the product of the two wave functions, $\ket{\Phi_A+\Phi_B} = \ket{\Phi_A}\ket{\Phi_B}$. That is because real physical systems are cluster decomposable. A completely uncorrelated Slater determinant is a cluster decomposable wave function. 
The most general way to include spin-isospin dependent correlations to a Slater determinant, while maintaining the cluster decomposability of the system, is to use an exponential of the correlation operators.
\begin{equation}
   \ket{\psi_T} = \prod\limits_{i<j}f_c(r_{ij}) e^{\sum\limits_p\fpij\Opij} \ket{\phi}
   \label{equ:exppsi}
\end{equation}
The calculation of this trial wave function grows exponentially with particle number, however a symmetrized product of pair operators has nearly the same form at second order, and exactly the same form in the independent pair condition, which we will describe in more detail shortly. The symmetrized product wave function is
\begin{equation}
   \ket{\psi_T} = \left[\prod\limits_{i<j}f_c(r_{ij})\right] \left[\mathcal{S}\prod\limits_{i<j}\left(1+\sum\limits_p\fpij\Opij\right)\right] \ket{\phi},
   \label{equ:prodpsi}
\end{equation}
where the symmetrization operator $\mathcal{S}$ is
\begin{equation}
   \mathcal{S}\ket{R} = \frac{1}{N!} \sum\limits_{\ket{R^p}} \ket{R^p},
\end{equation}
and $\ket{R^p}$ are all permutations of the particles coordinates. If you expand this to second order you can write this as
\begin{equation}
   \ket{\psi_T} = \left[\prod\limits_{i<j}\f{c}{ij}\right] \left[1+\fOpij+\frac{1}{2}\fOpij\fOqklquad\right] \ket{\phi}.
\end{equation}
Notice that the quadratic correlations include all sets of pairs except those with the exact same two particles. The exponential wave function in equation \ref{equ:exppsi} has these terms, with a different factor. Also notice that the quadratic terms are symmetrized by summing over each $ij$ pair and each $kl$ pair and then dividing by two. If instead of taking all possible quadratic pairs, only sets of pairs where the same particle does not appear twice are taken, this is called the independent pair correlation.
\begin{equation}
   \ket{\psi_T} = \left[\prod\limits_{i<j}\f{c}{ij}\right] \left[1+\fOpij+\fOpij\fOqklip\right] \ket{\phi}.
\end{equation}
Here the sum over independent pairs means that no particle is included in both pairs $ij$ and $kl$, and you do not get the symmetric pair terms, i.e. if you have the set of pairs $ijkl=1234$, you do not include $3412$. This is because the operators in the quadratic terms commute and do not need to be symmetrized. We have done calculations with both of these wave functions and compared the results to calculations done with the linear wave function
\begin{equation}
   \ket{\psi_T} = \left[\prod\limits_{i<j}f_c(r_{ij})\right] \left[1+\fOpij\right] \ket{\phi}.
\end{equation}
We have added a variational parameter to the quadratic terms, but for most of the results shown here that parameter was set to one, however we do show that the results improve as the variational parameter is optimized.











\if false
This is an infinite sum, and is impossible to calculate directly on a computer and so if we then assume that the correlations are small, we can then expand the exponential to first order getting
\begin{equation}
   \ket{\psi_T} = \prod\limits_{i<j}f_c(r_{ij}) \left(1+\sum\limits_p\fpij\Opij\right) \ket{\phi}.
   \label{equ:fullprod}
\end{equation}
We further expand this and truncate this product to first order, keeping only linear terms in the expansion. To improve this estimation we have included the quadratic terms from this expansion into the calculation, as well as a subset of the quadratic terms called independent pair terms. I'll show this expansion with the example of $A=3$ for convenience. Expanding equation \ref{equ:fullprod} we get
\begin{equation}
%\begin{align}
\begin{split}
   &\f{c}{12}\left(1+\fO{p}{12}\right)\f{c}{13}\left(1+\fO{p}{13}\right)\f{c}{23}\left(1+\fO{p}{23}\right) \\
   &~~=\f{c}{12}\f{c}{13}\f{c}{23}\left(1+\fO{p}{12}+\fO{p}{13}+\fO{p}{23}\right. \\
   &~~~~~+\fO{p}{12}\fO{q}{13}+\fO{p}{12}\fO{q}{23} \\
   &~~~~~+\left.\fO{p}{13}\fO{q}{23}\right) \\
   &~~= \left[\prod\limits_{i<j}\f{c}{ij}\right]\left[1+\fOpij+\frac{1}{2}\fOpij\fOqklquad + \ldots \right].
\end{split}
%\end{align}
\end{equation}
In the example with $A=3$ we only get up to triplet terms, of which there is only one, but for larger values of $A$ there are many more terms. With previous calculations we have approximated this by only taking up to the linear term,
\begin{equation}
   \ket{\psi_T} = \left[\prod\limits_{i<j}f_c(r_{ij})\right] \left[1+\fOpij\right] \ket{\phi}.
\end{equation}
We have included the quadratic correlations above, but we have also done a calculation with a subset of these correlations called independent pair terms. These independent pair terms only include pairs where there is no common particle among the two sets of pairs. Calculations were done with these independent pairs because we believed that they would capture most of the physics of the quadratic terms. \red{Look up nuclear matter calculations with FHNC by John Owen around 1980, like you have here.}
\begin{equation}
   \ket{\psi_T} = \left[\prod\limits_{i<j}f_c(r_{ij})\right] \left[1+\fOpij\left(1 + \fOqklip\right)\right] \ket{\phi}
\end{equation}
The sum $\sum\limits_{\mathrm{k<l,ip}}$ is a sum over all of the $kl$ pairs that do not have a particle that matches either the $i^\mathrm{th}$ or the $j^\mathrm{th}$ particle. This is the independent pair sum.
\fi


%\section{Potential}
As mentioned before, solving for the true nucleon-nucleon interaction from QCD is a difficult task. As a result, we often use phenomenological potentials such as Argonne $v_{18}$ for two-body forces or the Urbana or Illinois potentials for three-body forces.


\section{Results}
I have calculated binding energies with and without the independent pair correlations and compared the results. In both cases we have used the v6 potential and the same operators for the correlations. In each case the weights for each operator was determined variationally. Calculations were done for systems, $^4$He and $^{16}$O and the binding energies are reported in table \ref{tab:indpairresults} with and without independent pairs correlations and compared to the experimental value.

\begin{table}[h!]
   \centering
   \caption{Binding energies in MeV for $^4$He and $^{16}$O as calculated with and without independent pair correlations (IPC) compared to experimental energies.}
   \label{tab:indpairresults}
   \begin{tabular}{ccccc}
      \hline \hline
%       & Without IPC & With IPC & Expt.\\
       & Linear & IndPair & Quadratic & Expt.\\
      \hline
      $^4$He & -27.0(3) & -26.3(3) & -28.5(2) & -28.295\\
      $^{16}$O & -114(3) & -132(3) & -143.3(3) & -127.619\\
      \hline \hline
   \end{tabular}
\end{table}

I have also calculated the binding energy per nucleon of symmetric nuclear matter with density $\rho=0.16$fm$^{-1}$ of 28 particles with periodic boundary conditions. The energy per nucleon was -14.3(2) MeV without the independent pair correlations and -16.6(2) MeV with the independent pair correlations.

\begin{figure}[h!]
   \centering
   \includegraphics[width=0.5\textwidth]{energies.eps}
   \caption{Binding energies for ${}^4$He and ${}^{16}$O as calculated with linear, independent pair, and quadratic correlations. Also, the energy per nucleon of symmetric nuclear matter as calculated from 28 particles in a periodic box. All calculations are compared to their expected values.}
   \label{fig:energies}
\end{figure}


\section*{Conclusion}
I conclusion, I have learned how to implement VMC and DMC methods and have applied these methods to the quantum harmonic oscillator as well as a similar $x^4$ potential. The ground state energies and wave functions that I obtained were close to the exact solutions for the harmonic oscillator. I have also learned the basics of the AFDMC method and have being working on understanding the existing code. Further work will include the addition of the extra terms in the trial wave function correlations. These additions will then be compared to the existing code for a variety of nuclear systems.


\clearpage
\bibliographystyle{unsrt}
\bibliography{../../papers/references}

\end{document}
