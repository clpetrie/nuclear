\section{Conclusion and Outlook}
As mentioned before, one of the most important parts of the calculation is to have a good estimate for the trial wave function that is possible and inexpensive to calculate. So far we have implemented and tested a method for improving the trial wave function used in AFDMC calculation of nuclei and nuclear matter. We have done this by including independent pair and the full quadratic terms in the correlation operator which currently only has linear terms. The addition of these optimized terms has caused the energy of ${}^{4}$He, ${}^{16}$O and SNM to decrease.

Another way to improve the trial wave function is to include more terms in the expansion of the product correlation operator in Eq.~\ref{equ:prodpsi}. As we have learned from the independent pair expansion that we implemented, this would get computationally intractable as the number of particles gets large. Another alternative is to rewrite the correlation operator in terms of a Hubbard-Stratanovich transformation as is done with the spin-isospin dependent part of the propagator in Sec.~\ref{sec:AFDMC}. Future work will be done to handle the correlation operators in this way. We will also be searching for additional ways to improve the trial wave function.

With an improved trial wave function we will be using AFDMC to investigate some interesting aspects of nuclear physics and nuclear astrophysics. For example, we will be investigating the formation of deuteron and alpha particle clusters at various densities in mostly neutron matter. Many types of clustering in physics deal with two particles at a time. Alpha particles are thus a special case of four fermion clustering due to the two additional isospin degrees of freedom in addition to the two spin states for spin 1/2 particles. Work has been done to show the light clustering ($A\leq4$) and condensation of particles in nuclear systems, \cite{schuck2007,schuck2013_1,schuck2013_2} including neutron stars \cite{avancini2010,avancini2012,raduta2014}. We plan to show that the AFDMC method, with an improved nuclear wave function can be used to study properties of light clustering in nuclear systems as well. One way we will be doing this will be to do AFDMC calculation with 14 neutrons in a periodic box with the addition of 2 protons at a variety of densities to observe the formation of an alpha particle in neutron matter. Once these clusters are observed with AFDMC the density can be varied to determine the density at which the clusters dissolve as in Ref.\cite{avancini2012}. This work will be especially applicable to the study of alpha formation in neutron stars.

In conclusion, we have done AFDMC calculations to calculate the energy of ${}^4$He and ${}^{16}$O as well as the energy per particle of SNM. We have done these calculations with both linear correlation terms, linear plus independent pair correlations and linear plus all of the quadratic correlations in the trial wave function. In order to maintain the cluster decomposability of the trail wave function, which is lost with either the linear or independent pair correlations, we will be using the Hubbard-Stratanovich transformation to sample the exponential spin-isospin dependent correlations. This is analogous to the sampling of spin-isospin states in the propagator of AFDMC. We will also be looking for additional ways to improve the trial wave function. We will then be applying these calculations to other interesting nuclear systems such as the clustering of alpha particles in neutron matter.
