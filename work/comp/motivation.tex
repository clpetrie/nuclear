\section{Motivation}
Understanding the interactions between nucleons has been a lengthy and difficult task to pursue for science. However, a thorough understanding of these interactions would give us great insight into the structure of nuclei, the r-process by which nucleosynthesis occurs in supernovae, as well as the formation of neutron stars \cite{lattimer2001,lattimer2004,stone2003,douchin2001,heiselberg2000}. Both of these areas of research are difficult to probe experimentally and would benefit greatly from good theoretical predictions to guide experiments in useful and exciting directions.

Despite the difficulty, science has been making continuous steps towards understanding these interactions. In 1935 Hideki Yukawa proposed the idea that this interaction, called the strong force, was governed by quanta or exchange particles \cite{yukawa1935}, which were later called pions \cite{lattes1947}. He proposed what is now called the Yukawa potential, which is still used in modified form in many nuclear models today, to describe these interactions. The range of the force proposed by Yukawa was inversely proportional to the mass of the pion, and the strength of the force was based only on the distance separating the particles. Today we often use potentials that depend on the separation distance between the particles, but also their relative spins and isospins. These interactions can be quite complicated, making a true understanding of the strong force difficult to achieve.

Currently Quantum Chromodynamics (QCD) is believed to be the correct underlying theory to describe the strong force \cite{kisslinger2016}. However, this theory currently cannot be solved for nuclear particles, although some progress is being made. For example, calculations of n-n scattering have been done but they return pion masses that are too large, $m_\pi \sim 450$ MeV \cite{savage2016}. Instead approximate methods often based on effective field theories are used. The development of low energy effective field theories would provide useful insights into the nuclear force, even if we had low energy solutions to QCD.

A set of approximate methods called basis set methods are often used to solve for properties of nuclei. Some of these methods are the no core shell model \cite{navratil2009,barrett2013}, the coupled-cluster method \cite{hagen2014}, and the self-consistent Green's function method \cite{dickhoff2004,soma2014}. For these methods the wave function of the nuclear system is written in terms of a truncated basis, often a harmonic oscillator basis. The momentum cutoff of the basis needs to be higher than the important momenta of the interaction that is being used, in order to do calculations in momentum space. This means however that calculations with sharp potentials (like local hard wall potentials) are difficult to do with basis set methods. They do employ techniques such as Similarity Renormalization Group \cite{hergert2016} to soften these types of interaction. This allows them to decrease the number of basis functions used. One of the advantages of basis set methods is that they can use local and non-local, i.e. velocity dependent, potentials. The Quantum Monte Carlo \cite{carlson2015} methods, which we are using in this work, complement these basis set methods. Although they are currently limited to mostly local potentials\footnote{Currently, interactions that are linear in the momentum can be used. Higher order terms are treated perturbatively.} \cite{lynn2012}, they can converge for a wide variety of local Hamiltonians. Also, Quantum Monte Carlo methods do not have the momentum cutoff limits or the poor scaling with basis set size of the basis set methods.

Using these approximate methods to study nuclear systems will allow us to better understand the interactions between nuclei and ultimately will advance our understanding of many important processes in the universe.
