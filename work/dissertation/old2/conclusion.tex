\chapter{Conclusion}
\section{Summary}
Quantum Monte Carlo is a highly accurate method for studying the formation and structure of a variety of quantum systems including nuclei and neutron stars. Realistic models based on theoretical first principles, such as the newly developed models based on $\chi$EFT can be used with GFMC and AFDMC to study both the microscopic and macroscopic properties of nuclear systems. The GFMC method has been shown to be highly accurate, but is limited by the explicit sums over spin and isospin states. The AFDMC method has been shown to be an effective alternative to GFMC which allows for the calculation of much larger systems, up to $A=40$ for nuclei and $A>100$ for nuclear matter calculations but is limited to much simpler trial wave functions. All QMC methods depend heavily on an accurate trial wave function for their accuracy and convergence and so this is a major limitation to AFDMC.

The antisymmetric part of the wave function that has been used with AFDMC is typically a Slater determinant or a Pfaffian of single particle or pairing wave functions respectively. A simple set of Jastrow spin-isospin independent correlations along with a truncated version of the physical spin-isospin dependent correlations has been used in the past to correlate the wave function. The spin-isospin dependent correlations had been truncated at linear order in the pair correlations, $\mathcal{O}_{ij}$, in the past. I improved upon the linear correlations by adding quadratic correlations $\mathcal{O}_{ij}\mathcal{O}_{kl}$ in two different ways. The first method expanded the symmetrized product wave function \ref{equ:prodpsi}, which fully obeys cluster decomposition, to include all possible quadratic operators. The second approximation I made was identical to the first except that pairs including the same particle twice were not included. This is called the independent pair approximation.

I applied both quadratic approximations to study the doubly-closed shell nuclei $^4$He, $^{16}$O, and $^{40}$Ca as well a symmetric nuclear matter with 28 nucleons in a periodic box. I was able to show that the two approximations for the quadratic correlations were identical for all systems to within statistical errors. This indicates that the independent pair approximation captures most of the relevant physics. The quadratic correlations provided a significant decrease in energy for all of the larger systems. The alpha particle has strong symmetries that are well described by very simple wave functions as shown in Figure\ref{fig:energy_jaslin}. As a result, neither the linear nor the addition of the quadratic correlations changed the energy of the alpha particle by a measurable amount. I have also used this to study nuclear matter at saturation and found that, though the quadratic correlations are closer to known saturation properties, they do not improved the wave function enough to truly fit saturation.

I have also applied the quadratic correlations to the study of low density neutron and nearly neutron matter. This has application to the crust of neutron stars, specifically the density at which nuclei begin to dissolve into neutron rich nuclear matter. QMC has been instrumental in studying both low and high density neutron matter, applying to both nuclear properties as well as neutron stars. The mass-radius relation has been studying using the nuclear equation of state around saturation density as calculated by QMC. I have studied low density neutron matter in order to study the formation and dissolution of alpha particles in nearly neutron matter. I have done this by calculating the energy of 14 neutrons in a box with periodic boundary conditions and then adding 2 protons to the box. I have estimated what I have called the alpha particle energy, which is the energy of two protons and two neutrons that form an alpha particle at low densities. The binding energy of this alpha particle is much higher than that of a true alpha particle. I have thus concluded that the interactions with the remaining neutrons must be raising the energy of the system. Quadratic correlations were used to study this, and though the alpha particle energy did decrease with respect to calculations done with the linear correlations alone, there is still a difference of $\sim 4-6$ Mev. I have also compared the pair correlations function of the protons in this alpha particle to that calculated with continuum AFDMC. There was good agreement with these calculations indicating that at low densities an alpha particle is forming and then dissolving at higher densities.

AFDMC and other QMC methods are highly effective methods for studying nuclear many-body problems. I have improved the AFDMC method by improving the correlations in the trial wave function. Though the new quadratic correlations do provide an improvement in the energy of larger nuclear systems the computational cost is too high to be practical for large systems. This provides strong evidence that the previous spin-isospin correlations at linear order are not sufficient for AFDMC calculation of larger systems.

\section{Future Work}
Further improvements in both accuracy and efficiency of the trial wave function are needed, this is evident in my results. Any proposed trial wave function would need to be antisymmetric and obey cluster decomposition. One of the possible wave functions that obeys these properties is the exponentially correlated wave function in Equation~\ref{equ:exppsi}. As discussed before, this wave function is unfeasible to calculate directly. One possible extension to this research would be to use the Hubbard-Stratanovich transformation to sample these correlations in a random, Monte Carlo way. There is good preliminary evidence that these correlations could work well to describe larger systems in a computationally efficient way. There are some computational hurdles to overcome before this wave function is used in standard nuclear calculations. There is also the option of finding other efficient and accurate wave functions to be used the AFDMC that don't rely on either of the forms presented in this work.

The alpha particle formation calculations that I performed were done with 14 neutrons and 2 protons in a periodic box. These numbers were used because they fill plane wave shells, which makes calculations much easier with AFDMC. The next closed shell contained 38 neutrons with would have been much more expensive to calculate with the quadratic correlations. Being able to study the alpha formation with different neutron fractions is a possible future extension of this work. I briefly introduced the idea of twist boundary conditions which provide faster finite-size convergence as well as allowing for the use of different numbers of particles. An investigation of the effect of different neutron numbers may provide interesting results and insights into the dissolution of nuclei in neutron star crusts.

All of these calculations have been done with the two-body AV6$'$ potential. Previous calculations used a trial wave function that only correlated two particles at a time, yet the two-body AV6$'$ potential was improved significantly by correlations that correlated four nucleons at once. These higher-body correlations may be even more important when applied to three- or higher-body potentials. Another extension of this work would be to study the effect of these improved correlations on the Urbana and Illinois 3-nucleon potentials typically used with AFDMC as well as the new chiral potentials developed by $\chi$EFT.

QMC has made significant progress in recent years due to the advent of better algorithms, improved theories, and faster computers. Between this and the development of the new $\chi$EFT potentials, branching the highly accurate QMC methods with the underlying physics of QCD, this is an exciting time to be applying QMC to nuclear physics.
