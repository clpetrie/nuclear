\section{Ab-initio No-Core Shell Model}
To describe the No-Core Shell Model (NCSM) I am first going to briefly describe the Shell Model. The nuclear Shell Model is largely inspired by the atomic Shell Model. Both Shell Models rely on the Pauli exclusion principle, which requires that electrons stack up one at a time in each of the angular momentum states available. Since nucleons are spin 1/2, each angular momentum state can have two particles, one with spin up and the other with spin down. In the ground state the total spin of a particle is given by any unpaired particles. In the No-Core Shell Model all particles are allowed to contribute to the angular momentum properties of the nucleus, not just the unpaired particles.

