\section{Motivation}
Understanding the interactions between nucleons has been a lengthy and difficult task to persue for science. Despite the difficulty, science has been making continuous steps along the way. In 1935 Hideki Yukawa proposed the idea that this interaction, called the strong force, was governed by quanta or exchange particles \cite{yukawa1935}. These particles were later called gluons. From this idea was born the idea of the Yukawa potential, which is still used in modified form in many nuclear models today. The range of the force proposed by Yukawa was based on the mass of the exchange particle, and the strength of the force was based only on the distance separating the particles. Today we often use potentials that depend on the separation distance between the particles, but also their relative spins and isospins. These interactions can be quite complicated making a true understanding of the strong force difficult to achieve.

Currently it is believed that Quantum Chromodynamics (QCD) is the most correct theory to describe the strong force. However, due to asymptitic freedom at low energies, this theory becomes quite difficult to use and so other, approximate methods are often used to study the strong interaction. This is why we use Monte Carlo simulations to investigate different aspects of the strong interaction.

Continuing to better understand the interactions between nuclei will advance our understanding of many important processes in the universe.
