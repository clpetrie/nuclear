\section{Hartree Fock Method}
Hartree Fock (HF) is a method used to solve quantum many body problems. One can assume, as inspired by the independent particle model, that the wave function is an antisymmetrized product of unknown single particle orbitals,
\begin{equation}
   \psi_{\mathrm{HF}} = \mathcal{A} \prod\limits_{i=1}^A \phi_i(\mathbf{r}_i,s_i) = \frac{1}{A!} \mathrm{det}~\phi_i(\mathbf{r}_i,s_i),
\end{equation}
where the determinant is called a Slater determinant. This simple wave function assumes that the particles have no interaction other than some potential given by the average of all the other particles, as in mean field theory. HF is then used to determine the single-particle orbitals by minimizing the total energy, i.e. the expectation value of the Hamiltonian. We assume a Hamiltonian with kinetic energy for each particle and a two-body interactions potential of the form
\begin{equation}
   H=-\frac{\hbar^2}{2m}\sum\limits_{i=1}^A \nabla_i^2 + \sum\limits_{i<j} \hat{V}_{ij}.
\end{equation}
The variational principle is then used to minimize the expectation value of the energy, $E = \bra{\psi_{\mathrm{HF}}} H \ket{\psi{_\mathrm{HF}}}$. This is done by varying the single particle orbitals until a minimum in the energy is reached. The minimum energy is an upper bound on the true energy of the system, as given by the variational principle, and the set of orbitals that give the minimum energy gives an estimate to the true wave function of the system.

To do this you can first write the Hamiltonian in terms of single and pair particle operators.
\begin{equation}
   H = \sum\limits_{i=1}^A \hat{h}_i + \sum\limits_{i<j} \hat{V}_{ij}
\end{equation}
Then we will write the energy in terms of single particle states $\phi_i(x_i)$ and the operators above. You start with $E_\mathrm{HF}=\bra{\phi_\mathrm{HF}}H\ket{\phi_\mathrm{HF}}$, and after some algebra you can show that
\begin{equation}
   E_\mathrm{HF} = \sum\limits_{i} \bra{i}\sum\limits_j\hat{h}_j\ket{i} + \sum\limits_{i<j} \left(\bra{ii}\sum\limits_{k<l}V_{kl}\ket{jj}-\bra{ij}\sum\limits_{k<l}V_{kl}\ket{ji}\right).
\end{equation}
We then use the method of Lagrange's undetermined multipliers to minimize the above energy. If we allow the orbitals to vary, $\phi_i \rightarrow \phi_i + \delta\phi_i$, it can be shown that you get the hartree-fock equations, which are Schr\"odinger-like equations for the single-particle orbitals, $\phi_i$. \red{More on this later.}
