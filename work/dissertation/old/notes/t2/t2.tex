\documentclass[12pt]{article}
\usepackage[margin=1in]{geometry}
\usepackage{amssymb}
\usepackage{amsmath}
\usepackage{url}
\usepackage{bm}
\usepackage{color}

\title{Notes on $T^2$ operator}
\author{Cody L. Petrie}

\begin{document}
\maketitle

\section{$\exp(-\alpha T^2)$ operator to decrease $T^2$ violating terms}
The correlations violate $T^2$ and $T_z$. One solution to this is to multiply the correlations by a term $e^{-\alpha T^2}$ such that alpha is large enough to exponentially reduce the $T^2$ and $T_z$ breaking. This added piece to the correlations would take the form
\begin{equation}
   \exp{\left(-\alpha T^2\right)} = \exp{\left(-\alpha \sum\limits_\beta \sum\limits_{i,j} \tau_{i\beta}\tau_{j\beta}\right)}.
\end{equation}
The Hubbard-Stratanovich transformation can be used on these operators by writing them as an exponential of squared one-body operators.
\begin{equation}
   \exp{\left(-\alpha T^2\right)} = \exp{\left(-\alpha \sum\limits_\beta \left(\sum\limits_{j} \tau_{j\beta}\right)^2\right)}.
\end{equation}
In a standard AFDMC calculation the sum over $\beta$ would then be approximated by a product over $\beta$ where the commutator terms are small as long as the factor $\alpha$ is small. However, $\alpha$ must be large in order to eliminate $T^2$ breaking terms from the wave function and thus this approximation can't be used. Instead we have used the identity
\begin{equation}
   \exp{\left(\sum\limits_\beta A_\beta\right)} = \mathcal{S}\prod\limits_\beta \exp(A_\beta),
\end{equation}
where the symmetrization operator $\mathcal{S}=\frac{1}{N!}\sum\limits_\pi P_\pi$, permutes the Cartesian coordinates $\beta=xyz$. With this identity and the Hubbard-Stratanovich transformation we can write the correlations as
\begin{align}
   \exp{\left(-\alpha \sum\limits_\beta \left(\sum\limits_{j} \tau_{j\beta}\right)^2\right)}  &= \mathcal{S} \prod\limits_\beta \exp{\left(-\alpha \left(\sum\limits_{j} \tau_{j\beta}\right)^2\right)} \\
   &= \mathcal{S} \prod\limits_\beta \int dx_\beta \exp{\left(-x_\beta^2/2\right)}\exp{\left(i\sqrt{2\alpha}x_\beta \sum\limits_{j} \tau_{j\beta}\right)} \\
   &\approx \mathcal{S} \prod\limits_\beta \frac{1}{N}\sum\limits_{n=1}^N \exp{\left(i\sqrt{2\alpha}x_{n\beta} \sum\limits_{j} \tau_{j\beta}\right)},
\end{align}
where the sum over $n$ is a sum over the $N$ sampled configurations of the 3 auxiliary fields. The sum over $i$ can be brought out of the exponential as a product because the operators on different particles all commute. Also the symmetrization operator can be written as a sum over the $3!=6$ permutations of the $\beta$ coordinates giving us
\begin{equation}
   \frac{1}{6N} \prod\limits_j \sum\limits_{P(xyz)} \sum\limits_{n=1}^N \exp{\left(i\sqrt{2\alpha}x_{nx} \tau_{jx}\right)} \exp{\left(i\sqrt{2\alpha}x_{ny} \tau_{jy}\right)} \exp{\left(i\sqrt{2\alpha}x_{nz} \tau_{jz}\right)}
\end{equation}
The exponential operators on each particle look identical and can be written in a matrix representation as
\begin{align}
\exp\left(i\sqrt{2\alpha}x_{nx}\tau_{jx}\right) &=
\begin{pmatrix}
    \cos(a_{xn}) & 0 & i\sin(a_{xn}) & 0 \\
    0 & \cos(a_{xn}) & 0 & i\sin(a_{xn}) \\
    i\sin(a_{xn}) & 0 & \cos(a_{xn}) & 0 \\
    0 & i\sin(a_{xn}) & 0 & \cos(a_{xn}) \\
\end{pmatrix} \\
\exp\left(i\sqrt{2\alpha}x_{ny}\tau_{jy}\right) &=
\begin{pmatrix}
    \cos(a_{yn}) & 0 & \sin(a_{yn}) & 0 \\
    0 & \cos(a_y) & 0 & \sin(a_{yn}) \\
    -\sin(a_{yn}) & 0 & \cos(a_{yn}) & 0 \\
    0 & -\sin(a_{yn}) & 0 & \cos(a_{yn}) \\
\end{pmatrix} \\
\exp\left(i\sqrt{2\alpha}x_{nz}\tau_{jz}\right) &=
\begin{pmatrix}
    e^{ia_{zn}} & 0 & 0 & 0 \\
    0 & e^{ia_{zn}} & 0 & 0 \\
    0 & 0 & e^{-ia_{zn}} & 0 \\
    0 & 0 & 0 & e^{-ia_{zn}}
\end{pmatrix},
\end{align}
where $a_{xn}=\sqrt{2\alpha}x_{xn}$, $a_{yn}=\sqrt{2\alpha}x_{yn}$, $a_{zn}=\sqrt{2\alpha}x_{zn}$ and in our basis the iso-spin matricies are
\begin{equation}
\tau_x=
\begin{pmatrix}
    0 & 0 & 1 & 0 \\
    0 & 0 & 0 & 1 \\
    1 & 0 & 0 & 0 \\
    0 & 1 & 0 & 0
\end{pmatrix}
~~~~~~~\tau_y=
\begin{pmatrix}
    0 & 0 & -i & 0 \\
    0 & 0 & 0 & -i \\
    i & 0 & 0 & 0 \\
    0 & i & 0 & 0
\end{pmatrix}
~~~~~~~\tau_z=
\begin{pmatrix}
    1 & 0 & 0 & 0 \\
    0 & 1 & 0 & 0 \\
    0 & 0 & -1 & 0 \\
    0 & 0 & 0 & -1
\end{pmatrix}.
\end{equation}
Owing to the clean matrix representation of these operators the symmetrized product of exponential operators can be written as one matrix,
\begin{equation}
   \mathcal{M}_{jn} = \frac{1}{6} \sum\limits_{P(xyz)} =
\begin{pmatrix}
   A & 0 & B & 0 \\
   0 & A & 0 & B \\
   C & 0 & D & 0 \\
   0 & C & 0 & D
%    e^{ia_z}\cos(a_x)\sin(a_y) & 0 & \cos(a_z)\left(i\cos(a_y)\sin(a_x)+\cos(a_x)\sin(a_y)\right) & 0 \\
%    0 & e^{ia_z}\cos(a_x)\sin(a_y) & 0 & \cos(a_z)\left(i\cos(a_y)\sin(a_x)+\cos(a_x)\sin(a_y)\right) \\
%    \cos(a_z)\left(i\cos(a_y)\sin(a_x)-\cos(a_x)\sin(a_y)\right) & 0 & e^{-a_z}\cos(a_x)\cos(a_y) & 0 \\
%    0 & \cos(a_z)\left(i\cos(a_y)\sin(a_x)-\cos(a_x)\sin(a_y)\right) & 0 & e^{-a_z}\cos(a_x)\cos(a_y) \\
\end{pmatrix},
\end{equation}
where
\begin{align}
   A &= e^{ia_{zn}}\cos(a_{xn})\sin(a_{yn}) \\
   B &= \cos(a_{zn})\left(i\cos(a_{yn})\sin(a_{xn})+\cos(a_{xn})\sin(a_{yn})\right) \\
   C &= \cos(a_{zn})\left(i\cos(a_{yn})\sin(a_{xn})-\cos(a_{xn})\sin(a_{yn})\right) \\
   D &= e^{-a_{zn}}\cos(a_{xn})\cos(a_{yn}).
\end{align}
This matrix can then be build and operated on each of the particles.
\begin{equation} 
   \exp{\left(-\alpha T^2\right)} \approx \frac{1}{N} \prod\limits_{j=1}^A \sum\limits_{n=1}^N \mathcal{M}_{jn}
\end{equation}

\end{document}
