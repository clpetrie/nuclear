\section{Background and Motivation}
Nuclear physics sheds light on the extremes. From the structure and processes of atomic nuclei and hypernuclei to the formation and structure of some of the largest objects in the universe, neutron stars. One of the largest obstacles to these regimes stems from our incomplete knowledge of the nuclear interation. Once we settle on a possible interaction, the next obstacle is to solve for properties of many-body nuclear systems using the selected, and often complicated, interaction. Currently the popular choices for 2- and 3-body nuclear interactions come in two flavors, phenomenological and those based in Chiral Effective Field Theory ($\chi$EFT). There are a large number of methods that have been developed to solve the many-body nuclear problem, though I will be using the Auxiliary Field Diffusion Monte Carlo (AFDMC) method. Other notable methods are the basis set methods such as no core shell model \cite{navratil2009,barrett2013}, the coupled-cluster method \cite{hagen2014}, and the self-consistent Green's function method \cite{dickhoff2004,soma2014}. For these methods the wave function of the nuclear system is written in terms of a truncated basis, often a harmonic oscillator basis. The momentum cutoff of the basis needs to be higher than the important momenta of the interaction that is being used, in order to do calculations in momentum space. This means however that calculations with sharp potentials (like local hard wall potentials) are difficult to do with basis set methods. They do employ techniques such as Similarity Renormalization Group \cite{hergert2016} to soften these types of interaction. This allows them to decrease the number of basis functions used. One of the advantages of basis set methods is that they can use local and non-local, i.e. velocity dependent, potentials. The Quantum Monte Carlo (QMC) methods, which we are using in this work, complement these basis set methods. QMC methods are currently limited to mostly local potentials\footnote{Currently, interactions that are linear in the momentum can be used. Higher order terms are treated perturbatively.} \cite{lynn2012}, but can converge for a wide variety of local Hamiltonians. Also, Quantum Monte Carlo methods do not have the momentum cutoff limits or the poor scaling with basis set size of the basis set methods. \red{This is from my comp so maybe work it over a bit}

One of the most accurate QMC methods is the Green's Function Monte Carlo (GFMC) method, which has had good success calculating properties of light nuclei and nuclear matter using 2- and 3-body potentials as well as electroweak currents \cite{carlson2015}. GFMC has been used to calculate binding energies as well as excited states for nuclei up to $^{12}$C as well as the nuclear equation of state (EOS) which has been used to study the structure of neutron stars. Nuclear calculations using the GFMC method are limited due to the explicit sum over spin states when calculating expectation values. In 1999 Schmidt and Fantoni \cite{schmidt1999} proposed the AFDMC method which is practically identical to GFMC in its Monte Carlo sampling of spatial integrals, however AFDMC uses Monte Carlo to sample the spin-isospin sums as well.

\red{Add a bit about HF and how HF, GFMC, and AFDMC all use a similar form for the wave function}

In this study, for simplicity, I have only used the AV6' phenom. potential...

Despite the difficulty, science has been making continuous steps toward that understanding. In 1935 Hideki Yukawa proposed the idea that the nuclear interaction, called the strong force, was governed by quanta or exchange particles called pions \cite{yukawa1935}. From this idea came the Yukawa potential, which is still used in modified form in many nuclear models today. The range of the force proposed by Yukawa was based on the mass of the exchange particle, and the strength was based only on the distance separating the particles. Today we often use potentials that depend on the separation distance between particles, but also their relative spins and isospins. These interactions can be quite complicated making a true understanding of the strong force difficult to achieve.

Currently it is believed that Quantum Chromodynamics (QCD) is the most correct theory to describe the strong force. However, due to asymptotic freedom, at low energies this theory becomes quite difficult to use and so other, approximate methods are often used to study the strong interaction. We use Quantum Monte Carlo methods to investigate different aspects of the strong interaction.

Many approximate methods exist to solve the nuclear many body problem. Some of these include ??????????

Continuing to better understand the interactions between nuclei will advance our understanding of many important processes in the universe.
