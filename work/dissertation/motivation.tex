\section{Background and Motivation}
Nuclear physics sheds light on the extremes. From the structure and processes of atomic nuclei and hypernuclei to the formation and structure of some of the largest structures in the universe, neutron stars. One of the largest obstacles to these regimes is our lack of understanding of the interactions between nucleons and how to most effectively apply that knowledge to calculate properties of nuclear systems. Despite the difficulty, science has been making continuous steps toward that understanding. In 1935 Hideki Yukawa proposed the idea that the nuclear interaction, called the strong force, was governed by quanta or exchange particles called pions \cite{yukawa1935}. From this idea came the Yukawa potential, which is still used in modified form in many nuclear models today. The range of the force proposed by Yukawa was based on the mass of the exchange particle, and the strength was based only on the distance separating the particles. Today we often use potentials that depend on the separation distance between particles, but also their relative spins and isospins. These interactions can be quite complicated making a true understanding of the strong force difficult to achieve.

Currently it is believed that Quantum Chromodynamics (QCD) is the most correct theory to describe the strong force. However, due to asymptotic freedom, at low energies this theory becomes quite difficult to use and so other, approximate methods are often used to study the strong interaction. We use Quantum Monte Carlo methods to investigate different aspects of the strong interaction.

Many appriximate methods exist to solve the nuclear many body problem. Some of these include ??????????

Continuing to better understand the interactions between nuclei will advance our understanding of many important processes in the universe.
