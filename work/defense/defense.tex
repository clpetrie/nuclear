\documentclass{beamer}
\usepackage{pgfpages}
\usepackage[backend=bibtex]{biblatex}
\usepackage{multicol}
\setbeameroption{hide notes} % Only slides
%\setbeameroption{show only notes} % Only notes
%\setbeameroption{show notes on second screen=right} % Both
\bibliography{../../papers/references.bib}
\setbeamerfont{footnote}{size=\small}
%\AtEveryCitekey{\clearfield{title}}

%
% Choose how your presentation looks.
%
% For more themes, color themes and font themes, see:
% http://deic.uab.es/~iblanes/beamer_gallery/index_by_theme.html
%
\mode<presentation>
{
  \usetheme{Warsaw}      % or try Darmstadt, Madrid, Warsaw, ...
  \usecolortheme{default} % or try albatross, beaver, crane, ...
  \usefonttheme{default}  % or try serif, structurebold, ...
  \setbeamertemplate{navigation symbols}{}
  \setbeamertemplate{caption}[numbered]
} 

\usepackage[english]{babel}
%\usepackage[utf8x]{inputenc} %Doesn't play well with biblatex
\usepackage{amssymb}
\usepackage{bm}
\usepackage{color}
\usepackage{graphicx}

\newcommand{\red}[1]{{\color{red}{#1}}}
\newcommand{\ket}[1]{\left| #1 \right>}
\newcommand{\bra}[1]{\left< #1 \right|}
\newcommand{\braket}[2]{\left< #1 | #2 \right>}
\newcommand{\ketbra}[2]{\left| #1 \right> \left< #2 \right|}
\newcommand{\expect}[1]{\left< #1 \right>}
\newcommand{\fpij}{f_p(r_{ij})}
\newcommand{\vpij}{v_p(r_{ij})}
\newcommand{\Opij}{\mathcal{O}_{ij}^p}
\newcommand{\fOpij}{\sum\limits_{i<j}\sum\limits_p \fpij\Opij}
\newcommand{\fqkl}{f_q(r_{kl})}
\newcommand{\Oqkl}{\mathcal{O}_{kl}^q}
\newcommand{\fOqkl}{\sum\limits_{k<l}\sum\limits_q \fqkl\Oqkl}
\newcommand{\fOqklip}{\sum\limits_{k<l,\mathrm{ip}}\sum\limits_q \fqkl\Oqkl}
\newcommand{\fOqklquad}{\sum_{\substack{k<l\\ij \ne kl}}\sum\limits_q \fqkl\Oqkl}
\newcommand{\f}[2]{f_{#1}(r_{#2})}
\renewcommand{\O}[2]{\mathcal{O}_{#2}^{#1}}
\newcommand{\fO}[2]{\sum\limits_{#1} f_{#1}(r_{#2})\mathcal{O}_{#2}^{#1}}
\newcommand{\R}{\mathbf{R}}
\newcommand{\dt}{\Delta\tau}
\newcommand{\ti}{\bm{\tau}_i}
\newcommand{\tj}{\bm{\tau}_j}
\newcommand{\si}{\bm{\sigma}_i}
\newcommand{\sj}{\bm{\sigma}_j}
\newcommand{\sfont}{9}
\newcommand{\sspace}{10.2}

\title[Dissertation Defense]{{\large Dissertation Defense:}\\Improved Trial Wave Functions for Quantum Monte Carlo Calculations of Nuclear Systems and Their Applications}
\author[Cody L. Petrie]{Cody L. Petrie\\
Advisor: Kevin Schmidt}
\institute{Arizona State University}
%\date{}

\begin{document}

\begin{frame}
  \titlepage
\end{frame}

% Uncomment these lines for an automatically generated outline.
\begin{frame}{Outline}
  \tableofcontents
\end{frame}

% Commands to include a figure:
%\begin{figure}
%\includegraphics[width=\textwidth]{your-figure's-file-name}
%\caption{\label{fig:your-figure}Caption goes here.}
%\end{figure}

\section{Outline}
%\fontsize{\sfont}{\sspace}\selectfont
\begin{frame}{Outline}
Story: Why do we need to have a good trial wave function? Here are some options to improve the wave function. Here are the results we have gotten including some applications of the wave function.
\begin{itemize}
   \item Methods to solve the nuclear problem and why we use QMC
   \begin{itemize}\fontsize{\sfont}{\sspace}\selectfont
      \item VMC \red{June 15}
      \item DMC/AFDMC \red{June 15}
   \end{itemize}
   \item Trial wave function and why it's so important
   \begin{itemize}\fontsize{\sfont}{\sspace}\selectfont
      \item Slater Dets (and Pfaffians) \red{June 15}
      \item Jastrow and linear correlations \red{June 15}
      \item Quadratic correlations \red{June 30}
      \begin{itemize}
         \item Results \red{June 30}
      \end{itemize}
   \end{itemize}
   \item Other correlations
   \begin{itemize}\fontsize{\sfont}{\sspace}\selectfont
      \item Exponential correlations \red{June 30}
      \item Ale's correlations and $T^2$ fix \red{July 15}
   \end{itemize}
   \item Application to $\alpha$-clustering \red{July 15}
   \begin{itemize}\fontsize{\sfont}{\sspace}\selectfont
      \item Stefano's original results \red{July 15}
      \item Results with quadratic correlations \red{July 15}
   \end{itemize}
\end{itemize}
\end{frame}

\begin{frame}{Nuclear Many Body Problem}
   \begin{equation*}
      \left<H\right> = \bra{\Psi}H\ket{\Psi} = \int\Psi^*(\bm{R})H\Psi(\bm{R}) d\bm{R}
   \end{equation*}
   \begin{equation*}
      H = \sum\limits_{i=1}^A \frac{\bm{p}^2}{2m} + \sum\limits_{i<j} v_{ij} + \sum\limits_{i<j<k} V_{ijk} + \ldots
   \end{equation*}
   \begin{itemize}
      \item There are a number of ways to solve this problem.
      \begin{itemize}
         \item QCD
         \begin{itemize}
            \item Lattice QCD
         \end{itemize}
         \item No-core shell model
         \item Coupled-cluster
         \item Self consistent Green's function method
         \item Quantum Monte Carlo
      \end{itemize}
   \end{itemize}
   \red{Should I have a slide for each method or should I have some popup information about each and just describe them here?}
   \\\red{Can they be clumped into different styles and talked about together on slides?}
\end{frame}

%\note[itemize]{
%   \item test1
%   \item test2
%}

\end{document}
