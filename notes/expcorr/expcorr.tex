\documentclass[12pt]{extarticle}
\usepackage[margin=1in]{geometry}
\usepackage{amssymb}
\usepackage{amsmath}
\usepackage{url}
\usepackage{bm}
\usepackage{color}
\usepackage{cancel}

%My commands
\newcommand{\Oi}{\mathcal{O}_{i}}
\newcommand{\Oij}{\mathcal{O}_{ij}}
\newcommand{\Okl}{\mathcal{O}_{kl}}
\newcommand{\Oijp}{\mathcal{O}^p_{ij}}
\newcommand{\Oklp}{\mathcal{O}^p_{kl}}
\newcommand{\ket}[1]{\left| #1 \right>}
\newcommand{\bra}[1]{\left< #1 \right|}
\newcommand{\braket}[2]{\left< #1 | #2 \right>}
\newcommand{\ketbra}[2]{\left| #1 \right> \left< #2 \right|}
\newcommand{\ti}{\bm{\tau}_i}
\newcommand{\tj}{\bm{\tau}_j}
\newcommand{\si}{\bm{\sigma}_i}
\newcommand{\sj}{\bm{\sigma}_j}
\newcommand{\rij}{\hat{r}_{ij}}
\newcommand{\tia}{\tau_{i\alpha}}
\newcommand{\sia}{\sigma_{i\alpha}}
\newcommand{\sib}{\sigma_{i\beta}}
\newcommand{\tja}{\tau_{j\alpha}}
\newcommand{\tig}{\tau_{i\gamma}}
\newcommand{\sja}{\sigma_{j\alpha}}
\newcommand{\sjb}{\sigma_{j\beta}}
\newcommand{\tjg}{\tau_{j\gamma}}
\newcommand{\tij}{\ti \cdot \tj}
\newcommand{\sij}{\si \cdot \sj}
\newcommand{\mycolor}[1]{\textit{\textcolor{red}{#1}}}
\newcommand{\red}[1]{\textcolor{red}{#1}}
\newcommand{\longsi}{s_1, \ldots, s_{i-1} , s, s_{i+1}, \ldots, s_A}
\newcommand{\longsij}{s_1, \ldots, s_{i-1} , s, s_{i+1}, \ldots, s_{j-1}, s', s_{j+1}, \ldots ,s_A}
\newcommand{\Ot}{\mathcal{O}^\tau_{n\alpha}}
\newcommand{\Os}{\mathcal{O}^\sigma_{n}}
\newcommand{\Ost}{\mathcal{O}^{\sigma\tau}_{n\alpha}}
\newcommand{\detr}{\mathrm{det}}
\newcommand{\Aijt}{A^{\tau}_{i,j}}

\title{Notes on the Exponential Correlations}
\author{Cody L. Petrie}

\begin{document}
\maketitle

\section{Derive Exponential Wave Function}
The most general form of the fully correlated wave function is the exponentially correlated wave function given by
\begin{equation}
   \ket{\Psi_T} = \left[\prod\limits_{i<j}f_c(r_{ij})\right] e^{\sum\limits_{i<j,p}f_p(r_{ij})\Oijp} \ket{\Phi}
\end{equation}
The exponential is to maintain cluster decomposition of the wave function. The Jastrow spin-isospin independent correlations are handeled independent of my piece of the code and so I will ignore them here which effectively leaves me with
\begin{equation}
   \ket{\Psi_T} = e^{\sum\limits_{i<j,p}f_p(r_{ij})\Oijp} \ket{\Phi},
\end{equation}
where the operators in the sum are the standard $v6'$ operators, $\si\cdot\sj$, $\ti\cdot\tj$, $\si\cdot\sj \ti\cdot\tj$, $S_{ij}$ and $S_{ij} \ti\cdot\tj$, where $S_{ij} = 3\si\cdot\hat{r}_{ij}\sj\cdot\hat{r}_{ij}-\si\cdot\sj$. In an effort to write this an a sum of squared single particle operators (to be used with the Hubbard Stratanovich transformation) these operators can be written in the form
\begin{equation}
   \exp\left(\sum\limits_{i<j,p}f_p(r_{ij})\Oijp\right) = \exp\left(\frac{1}{2}\sum\limits_{i\alpha,j\beta} \sigma_{i\alpha}A^{\sigma}_{i\alpha,j\beta}\sigma_{j\beta}
      + \frac{1}{2}\sum\limits_{i\alpha,j\beta} \sigma_{i\alpha}A^{\sigma\tau}_{i\alpha,j\beta}\sigma_{j\beta}\ti\cdot\tj
      + \frac{1}{2}\sum\limits_{i,j} A^{\tau}_{i,j}\ti\cdot\tj\right).
\end{equation}
These matricies are simply another way to write the $f^p(r_{ij})$ function. Writting the operators in this form is discussed in more detail in appendix~\ref{app:hsprep}. In the code (in psicalc.f90) these matricies are actually called ``ftau(npart,npart)", ``fsig(3,npart,3,npart)" and ``fsigtau(3,npart,3,npart)". These matricies are zero when $i=j$, symmetric and can be written in terms of their eigenvalues and vectors.
\begin{align}
   &\sum\limits_{j\beta} A^{\sigma}_{i\alpha,j\beta}\psi^{\sigma}_{n,j\beta} = \lambda^{\sigma}_n\psi^{\sigma}_{n,i\alpha} \\
   &\sum\limits_{j\beta} A^{\sigma\tau}_{i\alpha,j\beta}\psi^{\sigma\tau}_{n,j\beta} = \lambda^{\sigma\tau}_n\psi^{\sigma\tau}_{n,i\alpha} \\
   &\sum\limits_{j} A^{\tau}_{i,j}\psi^{\tau}_{n,j} = \lambda^{\tau}_n\psi^{\tau}_{n,i}
\end{align}
The operators can then be written as
\begin{equation}
   \exp\left(\sum\limits_{i<j,p}f_p(r_{ij})\Oijp\right) = \exp\left(\frac{1}{2}\sum\limits_{n=1}^{3A} \left(O_{n}^{\sigma}\right)^2 \lambda_n^{\sigma}
      + \frac{1}{2}\sum\limits_{\alpha=1}^{3}\sum\limits_{n=1}^{3A} \left(O_{n\alpha}^{\sigma\tau}\right)^2 \lambda_n^{\sigma\tau}
      + \frac{1}{2}\sum\limits_{\alpha=1}^{3}\sum\limits_{n=1}^{A} \left(O_{n\alpha}^{\tau}\right)^2 \lambda_n^{\tau}\right),
\end{equation}
where the operators are given by
\begin{equation}
\begin{split}
   O_{n}^{\sigma} &= \sum\limits_{j,\beta} \sigma_{j,\beta}\psi_{n,j,\beta}^{\sigma} \\
   O_{n\alpha}^{\sigma\tau} &= \sum\limits_{j,\beta} \tau_{j,\alpha}\sigma_{j,\beta}\psi_{n,j,\beta}^{\sigma\tau} \\
   O_{n\alpha}^{\tau} &= \sum\limits_{j} \tau_{j,\alpha}\psi_{n,j}^{\tau}.
\end{split}
\end{equation}
Now this is ready to use with the Hubbard Stratanovich transformation
\begin{equation}
   e^{-\frac{1}{2}\lambda O^2} = \frac{1}{\sqrt{2\pi}} \int dx e^{-\frac{x^2}{2} + \sqrt{-\lambda}x O}.
\end{equation}
Writting this set of correlation operators in a more compact way,
\begin{equation}
    \exp\left(\sum\limits_{i<j,p}f_p(r_{ij})\Oijp\right) = \exp\left(\frac{1}{2}\sum\limits_{n=1}^{15A} \left(O_{n}\right)^2 \lambda_n^{\sigma}\right)
\end{equation}
allows for the application of the HS transformation, after breaking it into $15A$ exponentials and ignoring the commutation terms.
\begin{equation}
   \exp\left(\frac{1}{2}\sum\limits_{n=1}^{15A} \left(O_{n}\right)^2 \lambda_n^{\sigma}\right) = \prod\limits_{n=1}^{15A} \frac{1}{\sqrt{2\pi}}\int dx_n e^{-x_n^2/2}e^{\sqrt{\lambda_n}x_nO_n}.
\end{equation}

The auxiliary fields can then be drawn from the gaussian distribution, $\exp\left(-x_n^2/2\right)$ and the correlations can be written as follows.
\begin{equation}
   \Psi_T(R,S) = \bra{RS}\prod\limits_{n=1}^{15A} \frac{1}{N} \sum\limits_{\{x_n\}}^N\frac{1}{\sqrt{2\pi}}e^{\sqrt{\lambda_n}x_nO_n}\ket{\Phi}.
\end{equation}

\section{Exponential of an Operator}
To start I'll write down the exponential correlations in their full form. They look something like this
\begin{equation}
   \prod\limits_{n=1}^{15A} \frac{1}{\sqrt{2\pi}}\int dx_n e^{-x_n^2/2}e^{\sqrt{-\lambda_n}x_nO_n}.
\end{equation}
This is after the Hubbard-Staratanovich transformation has been applied to the correlations so that the correlations can be sampled from single-particle operators. If they are sampled they take a form something like
\begin{equation}
   \prod\limits_{n=1}^{15A} \frac{1}{\sqrt{2\pi}}\sum\limits_{x_n}^N\frac{1}{N}e^{\sqrt{-\lambda_n}x_nO_n}.
\end{equation}
It might also be important to note that we have added plus-minus sampling to this\ldots but I'll add that and the square root of the matrix stuff in later.

Now I will talk about each individual operator that we have in the correlations. The 15A operators are $\tau_{\alpha i}$ (3A), $\sigma_{\alpha i}$ (3A), and $\sigma_{\alpha i}\tau_{\beta j}$ (9A). Assuming the basis is $\left|p\uparrow,p\downarrow,n\uparrow,n\downarrow\right>$ and stored as a column vector we can write the operators in matrix form as follows.
\begin{equation*}
\tau_x=
\begin{pmatrix}
    0 & 0 & 1 & 0 \\
    0 & 0 & 0 & 1 \\
    1 & 0 & 0 & 0 \\
    0 & 1 & 0 & 0
\end{pmatrix}
~~~~~~~\tau_y=
\begin{pmatrix}
    0 & 0 & -i & 0 \\
    0 & 0 & 0 & -i \\
    i & 0 & 0 & 0 \\
    0 & i & 0 & 0
\end{pmatrix}
~~~~~~~\tau_z=
\begin{pmatrix}
    1 & 0 & 0 & 0 \\
    0 & 1 & 0 & 0 \\
    0 & 0 & -1 & 0 \\
    0 & 0 & 0 & -1
\end{pmatrix}
\end{equation*}
\begin{equation*}
\sigma_x=
\begin{pmatrix}
    0 & 1 & 0 & 0 \\
    1 & 0 & 0 & 0 \\
    0 & 0 & 0 & 1 \\
    0 & 0 & 1 & 0
\end{pmatrix}
~~~~~~~\sigma_y=
\begin{pmatrix}
    0 & -i & 0 & 0 \\
    i & 0 & 0 & 0 \\
    0 & 0 & 0 & -i \\
    0 & 0 & i & 0
\end{pmatrix}
~~~~~~~\sigma_z=
\begin{pmatrix}
    1 & 0 & 0 & 0 \\
    0 & -1 & 0 & 0 \\
    0 & 0 & 1 & 0 \\
    0 & 0 & 0 & -1
\end{pmatrix}
\end{equation*}
\begin{equation*}
\sigma_x\tau_x=
\begin{pmatrix}
    0 & 0 & 0 & 1 \\
    0 & 0 & 1 & 0 \\
    0 & 1 & 0 & 0 \\
    1 & 0 & 0 & 0
\end{pmatrix}
~~~~~~~\sigma_x\tau_y=
\begin{pmatrix}
    0 & 0 & 0 & -i \\
    0 & 0 & -i & 0 \\
    0 & i & 0 & 0 \\
    i & 0 & 0 & 0
\end{pmatrix}
~~~~~~~\sigma_x\tau_z=
\begin{pmatrix}
    0 & 1 & 0 & 0 \\
    1 & 0 & 0 & 0 \\
    0 & 0 & 0 & -1 \\
    0 & 0 & -1 & 0
\end{pmatrix}
\end{equation*}
\begin{equation*}
\sigma_y\tau_x=
\begin{pmatrix}
    0 & 0 & 0 & -i \\
    0 & 0 & i & 0 \\
    0 & -i & 0 & 0 \\
    i & 0 & 0 & 0
\end{pmatrix}
~~~~~~~\sigma_y\tau_y=
\begin{pmatrix}
    0 & 0 & 0 & -1 \\
    0 & 0 & 1 & 0 \\
    0 & 1 & 0 & 0 \\
    -1 & 0 & 0 & 0
\end{pmatrix}
~~~~~~~\sigma_y\tau_z=
\begin{pmatrix}
    0 & -i & 0 & 0 \\
    i & 0 & 0 & 0 \\
    0 & 0 & 0 & i \\
    0 & 0 & -i & 0
\end{pmatrix}
\end{equation*}
\begin{equation*}
\sigma_z\tau_x=
\begin{pmatrix}
    0 & 0 & 1 & 0 \\
    0 & 0 & 0 & -1 \\
    1 & 0 & 0 & 0 \\
    0 & -1 & 0 & 0
\end{pmatrix}
~~~~~~~\sigma_z\tau_y=
\begin{pmatrix}
    0 & 0 & -i & 0 \\
    0 & 0 & 0 & i \\
    i & 0 & 0 & 0 \\
    0 & -i & 0 & 0
\end{pmatrix}
~~~~~~~\sigma_z\tau_z=
\begin{pmatrix}
    1 & 0 & 0 & 0 \\
    0 & -1 & 0 & 0 \\
    0 & 0 & -1 & 0 \\
    0 & 0 & 0 & 1
\end{pmatrix}
\end{equation*}

Now those are easy enough to figure out if you just look at the pauli matricies on a simple spin-1/2 system. But what we really want is the exponential of these matricies with an extra factor up top that looks something like
\begin{equation}
   e^{\sqrt{-\lambda_n}x_nO_n}.
\end{equation}
I'm going to rewrite this with an $i$ in it as
\begin{equation}
   e^{i\gamma O_n},
\end{equation}
where the $\gamma=\sqrt{\lambda_n}x_n$, for each particular operator. In this way the exponentiated matricies can be written in terms of regular matricies. Plugging this into Mathematica I get the following.
\footnotesize
\begin{equation*}
e^{i\gamma\tau_x}=
\begin{pmatrix}
    \cos\gamma & 0 & i\sin\gamma & 0 \\
    0 & \cos\gamma & 0 & i\sin\gamma \\
    i\sin\gamma & 0 & \cos\gamma & 0 \\
    0 & i\sin\gamma & 0 & \cos\gamma
\end{pmatrix}
e^{i\gamma\tau_y}=
\begin{pmatrix}
    \cos\gamma & 0 & \sin\gamma & 0 \\
    0 & \cos\gamma & 0 & \sin\gamma \\
    -\sin\gamma & 0 & \cos\gamma & 0 \\
    0 & -\sin\gamma & 0 & \cos\gamma
\end{pmatrix}
e^{i\gamma\tau_z}=
\begin{pmatrix}
    e^{i\gamma} & 0 & 0 & 0 \\
    0 & e^{i\gamma} & 0 & 0 \\
    0 & 0 & e^{-i\gamma} & 0 \\
    0 & 0 & 0 & e^{-i\gamma}
\end{pmatrix}
\end{equation*}
\begin{equation*}
e^{i\gamma\sigma_x}=
\begin{pmatrix}
    \cos\gamma & i\sin\gamma & 0 & 0 \\
    i\sin\gamma & \cos\gamma & 0 & 0 \\
    0 & 0 & \cos\gamma & i\sin\gamma \\
    0 & 0 & i\sin\gamma & \cos\gamma
\end{pmatrix}
e^{i\gamma\sigma_y}=
\begin{pmatrix}
    \cos\gamma & \sin\gamma & 0 & 0 \\
    -\sin\gamma & \cos\gamma & 0 & 0 \\
    0 & 0 & \cos\gamma & \sin\gamma \\
    0 & 0 & -\sin\gamma & \cos\gamma
\end{pmatrix}
e^{i\gamma\sigma_z}=
\begin{pmatrix}
    e^{i\gamma} & 0 & 0 & 0 \\
    0 & e^{-i\gamma} & 0 & 0 \\
    0 & 0 & e^{i\gamma} & 0 \\
    0 & 0 & 0 & e^{-i\gamma}
\end{pmatrix}
\end{equation*}
\scriptsize
\begin{equation*}
e^{i\gamma\sigma_x\tau_x}=
\begin{pmatrix}
    \cos\gamma & 0 & 0 & i\sin\gamma \\
    0 & \cos\gamma & i\sin\gamma & 0 \\
    0 & i\sin\gamma & \cos\gamma & 0 \\
    i\sin\gamma & 0 & 0 & \cos\gamma
\end{pmatrix}
e^{i\gamma\sigma_x\tau_y}=
\begin{pmatrix}
    \cos\gamma & 0 & 0 & \sin\gamma \\
    0 & \cos\gamma & \sin\gamma & 0 \\
    0 & -\sin\gamma & \cos\gamma & 0 \\
    -\sin\gamma & 0 & 0 & \cos\gamma
\end{pmatrix}
e^{i\gamma\sigma_x\tau_z}=
\begin{pmatrix}
    \cos\gamma & i\sin\gamma & 0 & 0 \\
    i\sin\gamma & \cos\gamma & 0 & 0 \\
    0 & 0 & \cos\gamma & -i\sin\gamma \\
    0 & 0 & -i\sin\gamma & \cos\gamma
\end{pmatrix}
\end{equation*}
\begin{equation*}
e^{i\gamma\sigma_y\tau_x}=
\begin{pmatrix}
    \cos\gamma & 0 & 0 & \sin\gamma \\
    0 & \cos\gamma & -\sin\gamma & 0 \\
    0 & \sin\gamma & \cos\gamma & 0 \\
    -\sin\gamma & 0 & 0 & \cos\gamma
\end{pmatrix}
e^{i\gamma\sigma_y\tau_y}=
\begin{pmatrix}
    \cos\gamma & 0 & 0 & -i\sin\gamma \\
    0 & \cos\gamma & i\sin\gamma & 0 \\
    0 & i\sin\gamma & \cos\gamma & 0 \\
    -i\sin\gamma & 0 & 0 & \cos\gamma
\end{pmatrix}
e^{i\gamma\sigma_y\tau_z}=
\begin{pmatrix}
    \cos\gamma & \sin\gamma & 0 & 0 \\
    -\sin\gamma & \cos\gamma & 0 & 0 \\
    0 & 0 & \cos\gamma & -\sin\gamma \\
    0 & 0 & \sin\gamma & \cos\gamma
\end{pmatrix}
\end{equation*}
\begin{equation*}
e^{i\gamma\sigma_z\tau_x}=
\begin{pmatrix}
    \cos\gamma & 0 & i\sin\gamma & 0 \\
    0 & \cos\gamma & 0 & -i\sin\gamma \\
    i\sin\gamma & 0 & \cos\gamma & 0 \\
    0 & -i\sin\gamma & 0 & \cos\gamma
\end{pmatrix}
e^{i\gamma\sigma_z\tau_y}=
\begin{pmatrix}
    \cos\gamma & 0 & \sin\gamma & 0 \\
    0 & \cos\gamma & 0 & -\sin\gamma \\
    -\sin\gamma & 0 & \cos\gamma & 0 \\
    0 & \sin\gamma & 0 & \cos\gamma
\end{pmatrix}
e^{i\gamma\sigma_z\tau_z}=
\begin{pmatrix}
    e^{i\gamma} & 0 & 0 & 0 \\
    0 & e^{-i\gamma} & 0 & 0 \\
    0 & 0 & e^{-i\gamma} & 0 \\
    0 & 0 & 0 & e^{i\gamma}
\end{pmatrix}
\end{equation*}
\normalsize

\appendix
\section{Write potential as squared ops for AFDMC}
\label{app:hsprep}
The spin-isospin dependent potential can be written in the form
\begin{equation}
   V = \sum\limits_p\sum\limits_{i<j} u^p(\rij)\Oijp.
\end{equation}
I'm going to just write this out in terms of the simplest set of terms, the $\ti\cdot\tj$ terms.
\begin{align}
   V &= \sum\limits_{i<j} u^\tau(\rij)\Oijp \\
   &= \sum\limits_{i<j} u^\tau(\rij)\left(\tau_{ix}\tau_{jx}+\tau_{iy}\tau_{jy}+\tau_{iz}\tau_{jz}\right) \\
   &= \sum\limits_{i<j} u^\tau(\rij)\ti\cdot\tj \\
   &= \sum\limits_\alpha\sum\limits_{i<j} u^\tau(\rij)\tia\tja.
\end{align}
These can be rewritten in terms in a martix made of of the $u$ values. In the case of the $\tau$ operators this is simple because $\Aijt=u^\tau(\rij)$. If I rewrite the potential using this, and doing a full sum over all i and j and then dividing by 2 I get
\begin{equation}
   V = \frac{1}{2}\sum\limits_{\alpha,i,j} \Aijt\tia\tja
\end{equation}
Now I want to write this matrix in terms of it's eigenvalues and eigenvectors, which are defined as
\begin{equation}
   A^{\tau}\psi_n^\tau=\lambda_n^\tau\psi_n^\tau,
\end{equation}
or if you want to write them in the matrix multiplication form
\begin{equation}
   \sum\limits_j\Aijt\psi_{n,j}^\tau = \lambda_n^\tau\psi_{n,i}^\tau.
\end{equation}
Now I want to write out the $A$ matrix in terms of it's eigenvalues and eigenvectors. I do this using eigenvector decompositon. This is defined as
\begin{equation}
   A=Q\Lambda Q^{-1},
\end{equation}
where $Q$ is the matrix of eigenvectors (so $Q_{ab}=\psi_{ab}$, the $a^{th}$ eigenvector for the $b^{th}$ particle, for example), and $\Lambda$ is the diagonal matrix of eigenvalues (so $\Lambda_{ab}=\delta_{ab}\lambda_a$). I didn't prove this, but it's roughly believable when you consider the eigenvalue equation, and doing it for each component. Also, if $Q$ is a symmetric square matrix, which it is for us, then you can write $Q^{-1}=Q^T$i. What we have in the potential is $\Aijt$, so we need to write the eigenvector decomposition in terms of the $ij^{th}$ entry. This can be done by using the definitions of matrix multiplication.
\begin{equation}
   (A\psi)_i = \sum\limits_jA_{ij}\psi_j
\end{equation}
\begin{equation}
   (AB)_{ij}=\sum\limits_\alpha A_{i\alpha} B_{\alpha j}
\end{equation}
I'll now use these two equations to get $A_{ij}$ from the definition of eigenvalue decomposition.
\begin{align}
   A_{ij} &= \left(Q\Lambda Q^T\right)_{ij} \\
   &=\sum\limits_\alpha\sum\limits_\beta Q_{i\beta}\Lambda_{\beta\alpha}Q^T_{\alpha j} \\
   &=\sum\limits_\alpha\sum\limits_\beta Q_{i\beta}\delta_{\beta\alpha}\lambda_\alpha Q^T_{\alpha j} \\
   &=\sum\limits_\alpha \lambda_\alpha Q_{i\alpha}Q^T_{\alpha j} \\
   &=\sum\limits_\alpha \lambda_\alpha \psi_{i\alpha}\psi_{j\alpha}
\end{align}
Now plug this into the equation that we had earlier for the potential to get
\begin{align}
   V &= \frac{1}{2}\sum\limits_{\alpha,i,j} \Aijt\tia\tja \\
   &= \frac{1}{2}\sum\limits_{\alpha,i,j}\sum\limits_n\lambda_n^\tau\psi_{i,n}\psi_{j,n}\tia\tja \\
   &= \frac{1}{2}\sum\limits_\alpha\sum\limits_n\left(\Ot\right)^2\lambda^\tau_n,
   \label{equ:Vtau}
\end{align}
where
\begin{equation}
   \Ot = \sum\limits_i \tau_{i\alpha}\psi^\tau_n.
\end{equation}
A similar analysis can be done for the $\Os$ and $\Ost$ operators.

\bibliographystyle{unsrt}
\bibliography{../../papers/references.bib}

\end{document}
