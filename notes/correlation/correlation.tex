\documentclass[12pt]{extarticle}
\usepackage[margin=1in]{geometry}
\usepackage{amssymb}

%My commands
\newcommand{\Oop}{\mathcal{O}^p_{ij}}
\newcommand{\ket}[1]{\left| #1 \right>}
\newcommand{\bra}[1]{\left< #1 \right|}
\newcommand{\braket}[2]{\left< #1 | #2 \right>}

\title{Derivation of Trial Wave Function for Nuclear Monte Carlo}
\author{Cody Petrie}

\begin{document}
\maketitle

\section{Trial Wave Function}
To obtain the trial wave function for nuclear Monte Carlo there are two parts, the antisymmetric long range part ($\ket{\Phi}$) and the symmetric short range correlation operator ($\mathcal{F}$). That is
\begin{equation}
  \ket{\psi_T} = \mathcal{F} \ket{\Phi}.
\end{equation}
The antisymmetric part is made up of single particle orbitals in the form of a slater determinant. It is the symmetric correlation operator that I am not sure about. By reading other papers this is what I have understood. If you ignore three-body correlations then
\begin{equation}
  \mathcal{F} = \mathcal{S} \prod_{i<j} F_{ij},
\end{equation}
where
\begin{equation}
  F_{ij} = \sum_{p} f^p(r_{ij}) \Oop,
\end{equation}
where $\mathcal{S}$ is the symmetrization operator and $\Oop = 1, \sigma_i \cdot \sigma_j, 3\sigma_i \cdot \hat{r}_{ij} \sigma_j \cdot \hat{r}_{ij} - \sigma_i \cdot \sigma_j$ plus each of these terms times $\tau_i \cdot \tau_j$. To write this more like what other papers have written it I will pull the $1$ term out of the sum and let $f^1(r_{ij}) = f^c(r_{ij})$ since that term is a central term and I will call $u^p_{ij} = f^p(r_{ij})/f^1(r_{ij})$. This gives us
\begin{equation}
  F_{ij} = \sum_p f_c(r_{ij}) \left( 1 + \sum_p u^p(r_{ij}) \right).
\end{equation}
Putting all of these pieces together the trial wave function becomes
\begin{equation}
  \ket{\Phi_T} = \left[ \mathcal{S} \prod_{i<j} f_c(r_{ij}) \left(1+\sum_p u^p(r_{ij})\Oop\right) \right] \ket{\Phi}.
  \label{equ:mine}
\end{equation}
When we spoke the trial general trial wave function was supposed to be
\begin{equation}
  \ket{\Phi_T} = \left[ \prod_{i<j} f_c(r_{ij}) \left(1+\sum_{i<j}\sum_{p} u^p(r_{ij}) \Oop\right) \right]\ket{\Phi}.
  \label{equ:correct}
\end{equation}
Where the only differences are $\mathcal{S}$ in equation~\ref{equ:mine} and the double sum in equation ~\ref{equ:correct}. Does the double sum come from the symmetrization operator or am I just wrong with my derivation?

\end{document}
