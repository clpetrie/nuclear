\documentclass{beamer}
\usepackage{amsmath}
%\usepackage{beamerthemesplit} % new 
\usetheme{Madrid}
\usefonttheme[onlymath]{serif}
\setbeamertemplate{frametitle}[default][center] %center slide titles

%My commands
\newcommand{\ket}[1]{\left| #1 \right>}
\newcommand{\bra}[1]{\left< #1 \right|}
\newcommand{\braket}[2]{\left<\left. #1 \right| #2 \right>}
\newcommand{\ak}{a_k}
\newcommand{\akd}{a_k^\dagger}
\newcommand{\akb}{a_{\bar{k}}}
\newcommand{\akbd}{a_{\bar{k}}^\dagger}
\newcommand{\ai}{a_i}
\newcommand{\aid}{a_i^\dagger}
\newcommand{\aib}{a_{\bar{i}}}
\newcommand{\aibd}{a_{\bar{i}}^\dagger}
\newcommand{\alk}{\alpha_k}
\newcommand{\alkd}{\alpha_k^\dagger}
\newcommand{\alkb}{\alpha_{\bar{k}}}
\newcommand{\alkbd}{\alpha_{\bar{k}}^\dagger}
\newcommand{\ali}{\alpha_i}
\newcommand{\alid}{\alpha_i^\dagger}
\newcommand{\alib}{\alpha_{\bar{i}}}
\newcommand{\alibd}{\alpha_{\bar{i}}^\dagger}
\newcommand{\uk}{\mathrm{u}_k}
\newcommand{\vk}{\mathrm{v}_k}

\begin{document}
\title{Notes on Siemens Ch. 6}
\author{Cody Petrie} 
\date{\today} 

%start slides
\frame{\titlepage} 

\frame{\frametitle{Interactions Beyond the Mean Field}
\begin{itemize}
   \item The mean field approximation gives us basic features of nuclei. But now we're going to move beyond the mean field approximation.
   \item The first thing we are going to do is look at the pairing term to the binding energy (equation 4.3.2).
   \begin{equation}
      B_p = \frac{\left[(-1)^N+(-1)^Z\right]\delta}{A^{1/2}}
   \end{equation}
   \item This gives even-even nuclei a tighter binding energy. Also it turns out that the ground state of even-even nuclei have zero angular momentum.
   \item To explain these things we are going to go beyond the independent-particle motion (mean field).
\end{itemize}
}

\frame{\frametitle{Interactions Beyond the Mean Field}
\begin{itemize}
   \item  Add a perturbation to the mean field Hamiltonian ($H_R$ is called the residual interaction)
   \begin{equation}
      H = H_{MF} + H_R
   \end{equation}
   \begin{itemize}
      \item The eigenstates of $H_{MF}$ are Slater determinants.
   \end{itemize}
   \item One solution to this is to diagonalize $H_R$ in the $H_{MF}$ basis, but this requires large calculations.
   \item We are going to use other methods in this chapter. We will split (crudely) into long-range and short-range parts, and look at short-range parts here.
\end{itemize}
}

\frame{\frametitle{The $\delta$-Force}
\begin{itemize}
   \item Look at degenerate states of $H_{MF}$ because $H_R$ with have a decisive influence.
   \item Start with $H_{MF}$ in a full $j$ state and two identical nucleons in the next $j$ state.
   \begin{align}
      \psi^{nlj}_{JM}(1,2) &= \sum\limits_{m_1m_2} \braket{jm_1jm_2}{JM} \mathcal{A}\left[\Phi_{nljm_1}(1)\Phi_{nljm_2}(2)\right] \\
      \Phi_{nljm_1} &= \frac{1}{r} u_{nlj}(r) \sum\limits_{m,s} \braket{lm\frac{1}{2}s}{jm_1}Y_l^m(\theta,\phi)\chi_s
   \end{align}
   \item The shortest range for $H_R$ is a $\delta$-force.
   \begin{equation}
      H_R = V_0 \delta(\mathbf{r}_1 - \mathbf{r}_2)
   \end{equation}
\end{itemize}
}

\frame{\frametitle{The $\delta$-Force}
\begin{itemize}
   \item This Hamiltonian gives an energy
   \begin{align}
      E_R &= V_0 \int \psi^*_{JM}\delta(\mathbf{r}_1-\mathbf{r}_2)\psi_{JM}d^3\mathbf{r}_1 d^3\mathbf{r}_2 \\
      &= \frac{V_0\left[1+(-1)^J\right](2j+1)^2}{32\pi (2J+1)}\left|\braket{j,\frac{1}{2},j,-\frac{1}{2}}{J,0}\right|^2 \int\limits_0^\infty r^{-2}u^4_{nlj(r) dr}
   \end{align}
   \item Note here that $E_R$ vanishes for odd values of $J$. This means that two identical Fermi particles in the same j-shell can only be in even angular-momentum states.
   \item For an attractive force ($V_0<0$) the lowest energy has $J=0$ and the first excited state is $J=2$.
\end{itemize}
}

\frame{\frametitle{The $\delta$-Force}
\begin{itemize}
   \item For all $j> \frac{3}{2}$ the difference in energies of these two states is
   \begin{equation}
      \left|(E_2-E_0)/E_0\right| \approx \frac{3}{4}
   \end{equation}
   which is large as seen in figure 6.2 of the book.
   \item The two nucleons have their largest spatial overlap in this state ($J=0$).
   \item Thus an attractive $\delta$-interaction decreases the energy.
\end{itemize}
}

\frame{\frametitle{The Degenerate Pairing Model}
\begin{itemize}
   \item A main feature of the $\delta$-force that is maintained in the pairing force is that it only has non-zero matrix elements between time-reversed states. Also, they non-zero elements are all identital
   \begin{equation}
      \bra{jm_1\overline{jm_1}}V\ket{jm_2\overline{jm_2}} \equiv -G
   \end{equation}
   \item Let's use the basis states $j+\frac{1}{2} \equiv \Omega$. Now the Schr\"odinger equation becomes
   \begin{equation}
      -G\begin{pmatrix} 1 & 1 & \cdots & 1 \\ 1 & \hfill & \hfill & 1 \\ \vdots & \ddots & \vdots & \hfill \\ 1 & \hfill & \cdots & 1\end{pmatrix}
      \begin{pmatrix} x_1 \\ x_2 \\ \vdots \\ x_\Omega \end{pmatrix}
      = E\begin{pmatrix} x_1 \\ x_2 \\ \vdots \\ x_\Omega \end{pmatrix}
   \end{equation}
   \begin{equation}
      -G(x_1 + \cdots + x_\Omega) = Ex_1 = Ex_2 = \cdots = Ex_\Omega
   \end{equation}
\end{itemize}
}

\frame{\frametitle{The Degenerate Pairing Model}
\begin{equation}
   -G(x_1 + \cdots + x_\Omega) = Ex_1 = Ex_2 = \cdots = Ex_\Omega
\end{equation}
\begin{itemize}
   \item This has solutions
   \begin{equation}
      E = -G\Omega, \qquad \vec{x} = \frac{1}{\sqrt{\Omega}}(1,1,\cdots,1)
      \label{eq:groundenergy}
   \end{equation}
   and
   \begin{equation}
      E=0, \qquad x_1+x_2+\cdots+x_\Omega = 0.
      \label{eq:excitedenergy}
   \end{equation}
   \item Equation \ref{eq:groundenergy} refers to the $J=0$ state and the degerarate $J>0$ states have energy of equation \ref{eq:excitedenergy}.

\end{itemize}
}

\frame{\frametitle{The Degenerate Pairing Model}
\begin{itemize}
   \item Now assume we have $n$ particles in the j-shell ($n\le2\Omega$), and $p$ pairs of particles, i.e. $J=0$ states.
   \item Using equation \ref{eq:groundenergy} and the fact that $\Omega$ is the number of possible pairs we get
   \begin{equation}
      E(n,p)=-Gp\left(\Omega-n+p+1\right)
   \end{equation}
   \item Introduce \textit{seniority}, $S=n-2p$, the number of unpaired nucleons.
   \begin{align}
      E(n,S) &= -\frac{G}{4}(n-S)(2\Omega-n-S+2) \\
      E(n,0) &= -\frac{1}{4}Gn(2\Omega-n+2) \mathrm{,~ lowest} \\
      E(2,0) &= -G\Omega \mathrm{,~ what~ we~ got~ before} \\
      E(2\Omega,0) &= -Gn/2 \mathrm{,~ what~ you~ expect~ from~ mean~ field}
   \end{align}
\end{itemize}
}

\frame{\frametitle{The Degenerate Pairing Model}
\begin{itemize}
   \item The pairing force creates as many pairs of particles as possible.
   \item Even-even have zero spin.
   \item Odd numbers of nuclei, spin is determined by unpaired nucleon.
   \item Comparing odd-mass and even-mass nuclei we get,
   \begin{equation}
      E(2p+1,p)-E(2p,p) = Gp
   \end{equation}
   which seems to agree with experiment.
   \item For even-even nuclei, the lowest excited state is that of one broken pair, which has symmetric energy
   \begin{equation}
      E(2p,p-1)-E(2p,p)=G\Omega
   \end{equation}
\end{itemize}
}

\frame{\frametitle{General Pairing Theory}
\framesubtitle{The Model Hamiltonian}
\begin{itemize}
   \item For us to use degenerate perturbation theory we assume that the perturbed energy shifts should be small compared to the energy spacing. Since the shifts are $G\Omega$ this is not the case ($G$ is small but he says $G\Omega$ is much larger). So we have to do something else.
   \item We generalize our residual interaction and write it with creation and destruction operators.
   \begin{equation}
      H_R=-G\sum\limits_{i,k} {}^{'} \akbd \akd \ai \aib
   \end{equation}
   Where the primed sum means a truncated sum around the fermi energy $\left|\varepsilon_k - \mu\right| \le S$.
\end{itemize}
}

\frame{\frametitle{General Pairing Theory}
\framesubtitle{The Model Hamiltonian}
\begin{itemize}
   \item Since now we can't assume $H_R \ll H_{MF}$ we have to do some fancy footwork.
   \begin{equation}
      H = H_0+\delta H
   \end{equation}
   \begin{align}
      \label{equ:fullH0}
      H_0 &= \sum_k \varepsilon_k \left(\akd \ak + \akbd \akb\right) - \Delta\sum_k {}^{'} \left(\akbd \akd + \ak \akb\right) \\
      \delta H &= -G\sum_{i,k} {}^{'} \akbd \akd \ai \aib + \Delta\sum_k {}^{'} \left(\akbd \akd + \ak \akb\right)
   \end{align}
   \item We want eigenfunctions of both $H$ and $N$, but without $\delta H$ we can't have that.
   \begin{equation}
      N=\sum_k(\akd \ak + \akbd \akb)
   \end{equation}
   \item Thus we don't have conservation of particles and at most we can have a given average number particles. We account for this by modifying out Hamiltonian.
\end{itemize}
}

\frame{\frametitle{General Pairing Theory}
\framesubtitle{The Model Hamiltonian}
\begin{itemize}
   \item We then use the Lagrange multiplier $\mu$ (we will have a constraint on the expectation value of N) to write
   \begin{equation}
      H'=H-\mu N = H_0'+\delta H, \qquad H_0'=H_0-\mu N
      \label{equ:H0prime}
   \end{equation} 
   \item Minimizing the expectation value of $H_0$ ($\delta \left<H_0'\right>$) gives us
   \begin{equation}
      \mu=\frac{\delta \left<H_0\right>}{\delta \left<N\right>}
   \end{equation}
   \item This is the amound of energy that is needed to remove one nucleon, i.e. the chemical potential.
\end{itemize}
}

\frame{\frametitle{General Pairing Theory}
\framesubtitle{Solving the Unperturbed Hamiltonian}
\begin{itemize}
   \item We can diagonalize $H_0'$ exactly, we only have to do so for the interval $\left|\varepsilon_k-\mu\right| \le S$.
   \item Inside the interval we introduce new creation and annihilation operators.
   \begin{equation}
      \alk = \uk\ak+\vk\akbd \qquad \alkb=\uk\akb-\vk\akd
   \end{equation}
   \item This is called the Bogolyubov transformation and the variables $\uk$ and $\vk$ are used to minimize $\left<H_0'\right>$ with the normalization requirement
   \begin{equation}
      \uk^2+\vk^2=1,
   \end{equation}
   which ensures the same (anti)commutation relations between the $\alpha$'s as thos of the $a$'s.
   \begin{equation}
      \Big\{\ali,\alk\Big\} = \left\{\alid,\alkd\right\} = 0, \qquad \left\{\ali,\alkd\right\} = \delta_{ik}
   \end{equation}
\end{itemize}
}

\frame{\frametitle{General Pairing Theory}
\framesubtitle{Solving the Unperturbed Hamiltonian}
\begin{itemize}
   \item The transformation for $a$ is then
   \begin{equation}
      \ak=\uk\alk-\vk\alkbd, \qquad \akb=\uk\alkb+\vk\alkd.
   \end{equation}
   \item Plugging this into equations~\ref{equ:fullH0} and \ref{equ:H0prime} gives us
   \begin{equation}
      H_0'=\Omega_{gs}+\sum_kH_k^{(1)}\left(\alkd\alk+\alkbd\alkb\right) + \sum_k {}^{'} H_k^{(2)}\left(\alkbd\alkd+\alk\alkb\right)
   \end{equation}
   \begin{align}
      \Omega_{gs} &= 2\sum_k {}^{'} \left[(\varepsilon_k-\mu)\vk^2-\delta\uk\vk\right] \\
      H_k^{(1)} &= (\varepsilon_k-\mu)(\uk^2-\vk^2)+2\uk\vk\Delta \\
      H_k^{(2)} &= 2(\varepsilon_k-\mu)\uk\vk - (\uk^2-\vk^2)\Delta
   \end{align}
   \item Where for $\varepsilon_k-\mu>S, \uk=1, \vk=0$ and for $\varepsilon_k-\mu<-S, \uk=0, \vk=1$.
\end{itemize}
}

\frame{\frametitle{General Pairing Theory}
\framesubtitle{Solving the Unperturbed Hamiltonian}
\begin{itemize}
   \item To diagonalize $H_0'$ we require that $H_k^{(2)}$ be zero in the interval.
   \begin{equation}
      2(\varepsilon_k-\mu)\uk\vk=(\uk^2-\vk^2)\Delta
   \end{equation}
   \item This has the following solutions.
   \begin{align}
      \vk^2 &= \frac{1}{2}\left(1-\frac{\varepsilon_k-\mu}{\epsilon_k}\right) \le 1 \\
      \uk^2 &= \frac{1}{2}\left(1+\frac{\varepsilon_k-\mu}{\epsilon_k}\right) \le 1 \\
      \epsilon_k &= \sqrt{(\varepsilon_k-\mu)^2+\Delta^2}
   \end{align}
   \item With this we get
   \begin{equation}
      H_0' = \Omega_{gs}+\sum_k\epsilon_k\left(\alkd\alk+\alkbd\alkb\right)
   \end{equation}
\end{itemize}
}

\frame{\frametitle{General Pairing Theory}
\framesubtitle{Solving the Unperturbed Hamiltonian}
\begin{itemize}
   \item The BCS wave function is given by
   \begin{align}
      \alk\ket{gs}&=\alkb\ket{gs}=0 \\
      \ak\ket{vac}&=\akb\ket{vac}=0 \\
      \ket{gs} &= \prod_k\left(\uk+\vk\akbd\akd\right)\ket{vac}.
   \end{align}
   \item Again, since we don't have a fixed number of particles we can get the average number of particles, and the fluctuation in the number.
   \begin{align}
      N_0 = \bra{gs}N\ket{gs}=2\sum_k\vk^2=\sum_k\left(1-\frac{\varepsilon_k-\mu}{\epsilon_k}\right) \\
      \sigma^2 = \bra{gs}N^2\ket{gs}-(\bra{gs}N\ket{gs})^2 = 4\sum_k\uk^2\vk^2=\Delta^2\sum_k {}^{'}\frac{1}{\epsilon_k^2}
   \end{align}
\end{itemize}
}

\frame{\frametitle{General Pairing Theory}
\framesubtitle{Solving the Unperturbed Hamiltonian}
\begin{itemize}
   \item The energy gain compared to the normal energy is
   \begin{align}
      \Delta E &= \bra{gs}H_0'+\mu N\ket{gs}-2\sum_{\varepsilon_k<\mu} \varepsilon_k \\
      &= 2\sum_k\varepsilon_k\vk^2-\Delta^2\sum_k {}^{'}\frac{1}{\epsilon_k}-2\sum_{\varepsilon_k<\mu} \varepsilon_k
   \end{align}
   \item Looking at the BCS wave function we see that $\vk^2$ is the probability of one pair being in the original $\varepsilon_k$. This goes from 1 below $\mu$ to 0 above $\mu$.
   \item The system is now described by quasi-particles, of energy $\epsilon_k$, which are created by the $\alkd$ and $\alkbd$ operators. These quasi-particles are linear combinations of ``old" holes and ``old" particles.
\end{itemize}
}

\frame{\frametitle{General Pairing Theory}
\framesubtitle{Solving the Unperturbed Hamiltonian}
\begin{itemize}
   \item We can determine $\Delta$ by minimizing $\bra{gs}H-\mu N\ket{gs}$ with respect to $\Delta$.
   \item Doing this yields the condition
   \begin{equation}
      0=\Delta\left(1-\frac{1}{2}G\sum {}^{'}\frac{1}{\epsilon_k}\right),
   \end{equation}
   which has the solution $\Delta=0$, corresponding to the original Hartree-Fock solution with no pairing. For $\Delta \ne 0$ we get the condition
   \begin{equation}
      \frac{2}{G} = \sum {}^{'}\frac{1}{\epsilon_k}.
   \end{equation}
   \item This must be greater than
   \begin{equation}
      \sum \frac{1}{|\varepsilon_k-\mu|} = \frac{2}{G_c}.
   \end{equation}
   \item Thus to have non trivial solutions we must have $G\ge G_c$.
\end{itemize}
}

\frame{\frametitle{The Uniform Model}
\begin{itemize}
   \item If we assume $\Delta$ to be large compared to the single-particle energy spacings near the Fermi energy we can then take the level density, $g$, to be continuous. I we also assume that $S \gg \Delta$, and that $g(\varepsilon)$ is constant around $\mu$, we get
   \begin{align}
      \frac{2}{G}&=\sum {}^{'}\frac{1}{\epsilon_k} = \int\limits^{\mu+S}_{\mu-S}\frac{\frac{1}{2}g(\varepsilon)d\varepsilon}{\sqrt{(\varepsilon-\mu)^2+\Delta^2}} \approx \int\limits_0^S \frac{d\varepsilon}{\sqrt{\varepsilon^2+\Delta^2}} \\
      &=g(\mu) \ln\left(\frac{S}{\Delta}+\sqrt{1+\left(\frac{S}{\Delta}\right)^2}\right) \approx g(\mu) \ln \left(\frac{2S}{\Delta}\right),
   \end{align}
   which gives us
   \begin{equation}
      \Delta=2Se^{-2/Gg(\mu)}
   \end{equation}
\end{itemize}
}

\frame{\frametitle{The Uniform Model}
\begin{itemize}
   \item It will turn out that $\Delta$ is the physically relevant quantity, so in experiments it is reasonable to fix $\Delta$ and $g(\mu)$ to relate $S$ and $G$.
   \item The energy gain and the fluctuation of particle number can be estimated in similar ways to get
   \begin{align}
      \Delta E &\approx -\frac{1}{4}g(\mu)\Delta^2 \\
      \sigma^2 &\approx \frac{\pi}{2}g(\mu)\Delta.
   \end{align}
\end{itemize}
}

\frame{\frametitle{Relation to Experimental Information}
\begin{itemize}
   \item Let's now relate the Hamiltonian, $H_0'=\Omega_{gs}+\sum_k\epsilon_k\left(\alkd\alk+\alkbd\alkb\right)$ to experiment. The lowest excited state for even-even nuclei is the lowest possible two-quasiparticle excitation
   \begin{equation}
      2\epsilon_k=2\sqrt{(\varepsilon-\mu)^2+\Delta^2} \ge 2\Delta.
   \end{equation}
   This corresponds to the breaking of one pair.
   \item If you take an odd nucleus as an even-even core plus an extra quasiparticle, then you can use this for odd systems as well. For this the ground state energy is $\Omega_{gs} + \epsilon_{k_0}$, giving the lowest excited state an energy of $\epsilon_k-\epsilon_{k_0}$.
   \item This means that the binding energy for an odd mass nucleus is about $\epsilon_{k_0} \approx \Delta$ larger than for even-even nuclei. This is experimentally known from the liquid-drop model and is roughly
   \begin{equation}
      \Delta = \frac{2\delta}{\sqrt{A}}=\frac{12\mathrm{MeV}}{\sqrt{A}}
   \end{equation}
\end{itemize}
}

\frame{\frametitle{Relation to Experimental Information}
\begin{itemize}
   \item Figure 6.4 shows results for $\Delta$ values that match experiment pretty well, which is evidence that the strong force has two-body interactions similar to the pairing force.
   \item The author then goes on to give some numerical estimates of $g(\mu)$, $S$, $G$, and $\sigma$.
\end{itemize}
}

\end{document}
