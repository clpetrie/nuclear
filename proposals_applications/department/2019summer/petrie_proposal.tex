\documentclass[12pt]{article}
\usepackage[margin=1.00in]{geometry}
\usepackage{amssymb}
\usepackage{amsmath}
\usepackage{url}
\usepackage{bm}
\usepackage{color}
\usepackage{graphicx}

\newcommand{\red}[1]{{\color{red}{#1}}}
\newcommand{\blue}[1]{{\color{blue}{#1}}}

\pagenumbering{gobble} %remove page number

\title{\vspace{-2.0cm}Neutron Star Physics from Microscopic Nuclear Physics}
\author{Cody L. Petrie}
\date{}

\begin{document}
\maketitle

For the last five years I have worked and collaborated with others at ASU to produce quality learning experiences for students as well as producing meaningful research. At the end of this summer my time and service at ASU will culminate with my graduation. ASU has provided me with many useful opportunities to serve and to grow. I have been able to serve as a teaching assistant for a variety of classes, including the online based 113 and 122 physics labs under the direction of Darya Dolenko. I was sponsored by the physics department to be in instructor for the Clubes de Ciencia program whose goal is to inspire young scholars from Latin American countries (I served in Mexico). I have been well instructed by the physics curriculum at ASU, through which I obtained a 4.0 GPA. In addition to these teaching opportunities I have been well instructed on matters of physics research by Kevin Schmidt, my graduate committee, as well as collaborators that I have worked with both in and out of the country. I recently was able to present my research at the 4CS APS meeting and was awarded the Outstanding Graduate Student Presentation Award. I have also been nominated as a College of Liberal Arts and Sciences Student Leader for both the 2017 and 2018 years. My research has been published in the APS Physical Review C journal, and plan to begin the publishing process for my most recent work this summer.

I have been working with Kevin Schmidt and others developing improved nuclear trial wave functions to use with the Auxiliary Field Diffusion Monte Carlo method (AFDMC). The AFDMC method improves on the computational efficiency of previous similar methods such as Green's Function Monte Carlo, however, this improvement comes at the expense of the accuracy of the trial wave function. Previous AFDMC calculations used a very simple wave function with spin-isospin dependent correlations that only correlated two nucleons at a time\footnote{Stefano Gandolfi, et al., Phys. Rev. C 90 (2014).}. I expanded these correlations to include four-nucleon correlations as well. These additional correlations improved the trial wave function, which resulted in the significant improvement in binding energy calculations for nuclei up to $^{40}$Ca as well as nuclear matter\footnote{Diego Lonardoni, et al., Phys. Rev. C 97, 044318 (2018).}.

I have recently been using these correlations to study the clustering of nucleons in low density nuclear matter into nuclei such as the alpha particle. Specifically, I investigated nuclear matter that resembled conditions similar to those in the crust of neutron stars. At low enough densities nuclei can exist in neutron star crusts, however, as the density increases these nuclei dissolve into nearly neutron matter. I have been studying the formation and dissolution of alpha particles near the transition density. Previous efforts have been made but calculations with the previous correlations underbind the alpha particle, even at very low densities. Preliminary results have shown that the improved correlations provide some of the additional binding energy.

Over the summer I will complete this work and submit it for publication in addition to finishing and defending my dissertation. Upon completing my graduation requirements I will continue to work with Kevin and other collaborators to study other possible improvements to the trial wave function as well as study other interesting problems in nuclear physics.

\end{document}
